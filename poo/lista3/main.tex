\documentclass{article}
\usepackage{style}

\begin{document}

% University Logo, Course Name, Professor's Name, Date, Homework Title,
% Homework Number, Student Name, Student Identifier, Source Code
\cover{./UTFPR.png}{Fundamentos de Orientação à Objetos}{Leonardo Medeiros}{17/02/2025}
{Lista de Exercícios}{3}{Gabriel dos Santos Schmitz}{2487438}
{https://github.com/gabrielzschmitz/uni/tree/main/poo/lista3}

\newpage
\question{Escreva um programa C++ que use std::map}
\answer{
	\inputminted[frame=none, framesep=2mm, baselinestretch=1.2, bgcolor=whitesmoke,
		fontsize=\footnotesize, linenos]{Cpp}{./code/1.cpp}
}

\newpage
\question{Escreva um programa C++ que use std::multimap}
\answer{
	\inputminted[frame=none, framesep=2mm, baselinestretch=1.2, bgcolor=whitesmoke,
		fontsize=\footnotesize, linenos]{Cpp}{./code/2.cpp}
}

\newpage
\question{Escreva um programa C++ que use std::set}
\answer{
	\inputminted[frame=none, framesep=2mm, baselinestretch=1.2, bgcolor=whitesmoke,
		fontsize=\footnotesize, linenos]{Cpp}{./code/3.cpp}
}

\newpage
\question{Escreva um programa C++ que use std::set}
\answer{
	\inputminted[frame=none, framesep=2mm, baselinestretch=1.2, bgcolor=whitesmoke,
		fontsize=\footnotesize, linenos]{Cpp}{./code/4.cpp}
}

\newpage
\question{Escreva um programa C++ que use std::stack}
\answer{
	\inputminted[frame=none, framesep=2mm, baselinestretch=1.2, bgcolor=whitesmoke,
		fontsize=\footnotesize, linenos]{Cpp}{./code/5.cpp}
}

\newpage
\question{Escreva um programa C++ que use std::queue}
\answer{
	\inputminted[frame=none, framesep=2mm, baselinestretch=1.2, bgcolor=whitesmoke,
		fontsize=\footnotesize, linenos]{Cpp}{./code/6.cpp}
}

\newpage
\question{Escreva um programa C++ que use std::priority\_queue}
\answer{
	\inputminted[frame=none, framesep=2mm, baselinestretch=1.2, bgcolor=whitesmoke,
		fontsize=\footnotesize, linenos]{Cpp}{./code/7.cpp}
}

\section*{Compilação e Execução dos Exercícios}

Para compilar e executar os exercícios \( 1 - 7 \), siga os passos abaixo:

\begin{enumerate}
	\item \textbf{Torne o script \texttt{run\_tests.sh} executável}: \\ No
	      terminal, execute o seguinte comando para tornar o script executável:
	      \begin{minted}[frame=none, framesep=2mm, baselinestretch=1.2,
          bgcolor=whitesmoke, fontsize=\footnotesize, linenos]{Bash}
          $ chmod +x run_tests.sh
        \end{minted}

	\item \textbf{Execute o script para compilar e rodar os códigos}: \\ Em
	      seguida, execute o script para compilar e rodar os códigos dos exercícios:
	      \begin{minted}[frame=none, framesep=2mm, baselinestretch=0.2,
        bgcolor=whitesmoke, fontsize=\footnotesize, linenos]{Bash}
          $ ./run_tests.sh
        \end{minted}
\end{enumerate}

Isso deve compilar os arquivos com \texttt{g++} e rodar os testes
automaticamente. Se houver algum erro na compilação, o script irá mostrar as
mensagens de erro.

\end{document}
