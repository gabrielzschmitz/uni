\section{Atividades}

\question[Atividade 1.1]{Comparando integridade de arquivos de texto no Kali Linux}
\begin{figure}[H]
  \centering
  \includegraphics[width=1.0\textwidth]{./fig/fig1.1.png}
  \caption{Verificação de integridade entre arquivos de texto utilizando o
  comando \texttt{diff}}
  \label{fig:fig1.1}
\end{figure}
\answer{
Nesta atividade, utilizou-se o comando \texttt{diff} no Kali Linux para
verificar a integridade de arquivos de texto. Após a criação e cópia de um
arquivo, a comparação inicial não indicou diferenças. Em seguida, uma
modificação foi realizada no arquivo copiado, e o comando \texttt{diff}
identificou corretamente a alteração na primeira linha, demonstrando sua
eficácia na detecção de mudanças em arquivos de texto.
}

\newpage
\question[Atividade 1.2]{Comparando integridade de arquivos genéricos no Kali Linux}
\begin{figure}[H]
  \centering
  \includegraphics[width=1.0\textwidth]{./fig/fig1.2.png}
  \caption{Verificação de integridade entre arquivos binários utilizando o
  comando \texttt{cmp}}
  \label{fig:fig1.2}
\end{figure}
\answer{
Nesta atividade, utilizou-se o comando \texttt{cmp} no Kali Linux para
verificar a integridade de arquivos genéricos, como imagens. Inicialmente, dois
arquivos idênticos foram comparados sem que houvesse retorno, confirmando a
igualdade entre eles. Em seguida, após uma modificação em um dos arquivos, o
comando \texttt{cmp} identificou corretamente a diferença, indicando o byte em
que a alteração ocorreu, demonstrando sua eficácia na detecção de mudanças em
arquivos binários.
}

\newpage
\question[Atividade 1.3]{Verificando integridade de arquivos com função Hash no Kali Linux}
\begin{figure}[H]
  \centering
  \includegraphics[width=1.0\textwidth]{./fig/fig1.3.png}
  \caption{Verificação de integridade de arquivos utilizando função hash por
  meio do comando \texttt{md5sum}}
  \label{fig:fig1.3}
\end{figure}
\answer{
Nesta atividade, utilizou-se o comando \texttt{md5sum} no Kali Linux para
verificar a integridade de arquivos por meio de funções hash. Arquivos
idênticos apresentaram o mesmo valor de hash MD5, enquanto arquivos modificados
ou diferentes geraram hashes distintos. Dessa forma, foi demonstrada a eficácia
do uso de funções hash na identificação de alterações e na verificação de
integridade de arquivos.
}

\newpage
\question[Atividade 1.4]{Cifrando texto para proteger a confidencialidade no Kali Linux}
\begin{figure}[H]
  \centering
  \includegraphics[width=1.0\textwidth]{./fig/fig1.4.png}
  \caption{Aplicação das cifras ROT13 e César para cifrar e decifrar mensagens
  no Kali Linux}
  \label{fig:fig1.4}
\end{figure}
\answer{
Nesta atividade, foram aplicadas técnicas de cifragem para proteger a
confidencialidade de textos no Kali Linux. Inicialmente, utilizou-se o
algoritmo ROT13 para cifrar e decifrar uma mensagem de forma simétrica. Em
seguida, foi implementada a Cifra de César por meio de um script em Python,
permitindo a cifragem e a decifragem de textos com diferentes deslocamentos.
Assim, foi demonstrado o uso de cifras clássicas na proteção básica da
informação.
}

\newpage
\question[Atividade 1.5]{Controlando fluxo de trabalho via linha de comando no Kali Linux}
\begin{figure}[H]
  \centering
  \includegraphics[width=1.0\textwidth]{./fig/fig1.5.png}
  \caption{Gerenciamento de tarefas via linha de comando utilizando a
  ferramenta Taskwarrior}
  \label{fig:fig1.5}
\end{figure}
\answer{
Nesta atividade, utilizou-se a ferramenta Taskwarrior no Kali Linux para o
gerenciamento de tarefas via linha de comando. Foram criadas e listadas
tarefas, aplicadas marcações de urgência, definidos prazos e concluídas
atividades. Dessa forma, demonstrou-se o uso do Taskwarrior como uma solução
eficiente para o controle de fluxo de trabalho em ambientes baseados em
terminal.
}
