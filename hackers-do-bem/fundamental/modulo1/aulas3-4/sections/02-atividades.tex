\section{Atividades}

\question[Atividade 1.6]{Conhecendo a ferramenta de Phishing ShellPhish no Kali Linux}
\begin{figure}[H]
  \centering
  \includegraphics[width=1.0\textwidth]{./fig/fig1.6.png}
  \caption{Simulação de ataque de phishing utilizando a ferramenta
  ShellPhish no Kali Linux}
  \label{fig:fig1.6}
\end{figure}
\answer{
Nesta atividade, foi utilizada a ferramenta ShellPhish para simular um ataque
de phishing em ambiente controlado no Kali Linux. A aplicação foi configurada
para hospedar localmente uma página falsa de login do Facebook, permitindo a
captura de credenciais inseridas para fins de demonstração.

Após o acesso à página hospedada em \texttt{localhost}, as credenciais
digitadas foram registradas no terminal, evidenciando o funcionamento da
ferramenta e demonstrando, de forma prática, o princípio básico de ataques de
phishing voltados à captura de informações sensíveis.
}

\newpage
\question[Atividade 1.7]{Explorando a Ferramenta WHOIS no Kali Linux}
\begin{figure}[H]
  \centering
  \includegraphics[width=1.0\textwidth]{./fig/fig1.7.png}
  \caption{Consulta de informações públicas de domínios utilizando a ferramenta
  WHOIS no Kali Linux}
  \label{fig:fig1.7}
\end{figure}
\answer{
Nesta atividade, foi explorada a ferramenta \texttt{whois}, nativa do Kali
Linux, com o objetivo de coletar informações públicas associadas a registros de
domínios na Internet. Após a elevação de privilégios para superusuário
(\texttt{sudo -i}), foram realizadas consultas aos domínios \texttt{rnp.br},
\texttt{guanambi.ba.gov.br} e \texttt{itapaje.ce.gov.br}.

A análise das respostas permitiu identificar informações relevantes como:

\begin{itemize}
    \item Nome do domínio registrado (\texttt{domain});
    \item Entidade proprietária (\texttt{owner});
    \item Identificador do proprietário no Registro.br (\texttt{owner-c});
    \item Responsável técnico pelo domínio (\texttt{tech-c});
    \item Servidores de nomes (DNS) associados (\texttt{nserver});
    \item Datas de criação, alteração e status do domínio;
    \item Registros de segurança DNSSEC (\texttt{dsrecord} e
      \texttt{dsstatus}).
\end{itemize}

Observou-se que domínios governamentais estaduais e municipais utilizam
infraestruturas centralizadas de TI (como PRODEB e ETICE), evidenciando modelos
de gestão tecnológica compartilhada.
}

\newpage
\question[Atividade 1.8]{Explorando a Ferramenta de Engenharia Social Maltego no Kali Linux}
\begin{figure}[H]
  \centering
  \includegraphics[width=1.0\textwidth]{./fig/fig1.8.png}
  \caption{Configuração inicial e exploração da aba Transforms no Maltego Data
  Hub}
  \label{fig:fig1.8}
\end{figure}
\answer{
Nesta atividade, foi realizada a configuração inicial e exploração da
ferramenta \textit{Maltego}, disponível nativamente no Kali Linux, com foco na
utilização de recursos de \textit{OSINT} (Open Source Intelligence).

Inicialmente, foi criada uma conta gratuita no site oficial do Maltego. Em
seguida, na máquina virtual Kali Linux, o software foi iniciado pelo terminal
com o comando \texttt{maltego}. Durante o processo de ativação, foram
selecionadas as opções \textit{Maltego ID} e \textit{Online Activation},
realizado o login via navegador (Browser Login) e aceitos os termos de uso
(\textit{Data Sources T\&Cs}).

Após a conclusão da configuração, o ambiente principal do Maltego foi acessado.
Foi então explorada a aba \textbf{Transforms} e o \textbf{Maltego Data Hub},
onde estão disponíveis diversas integrações e fontes de dados OSINT, utilizadas
para coleta e correlação de informações públicas, como domínios, endereços IP,
e-mails, organizações e perfis digitais.
}

\newpage
\question[Atividade 1.9]{Reconhecimento Passivo OSINT com Maltego no Kali Linux}
\begin{figure}[H]
  \centering
  \includegraphics[width=1.0\textwidth]{./fig/fig1.9.png}
  \caption{Execução de Transforms no domínio ufc.br e identificação de
  servidores MX}
  \label{fig:fig1.9}
\end{figure}
\answer{
Nesta atividade, foi realizado reconhecimento passivo utilizando a ferramenta
\textit{Maltego}, com foco em técnicas de \textit{OSINT} (Open Source
Intelligence) aplicadas a um domínio institucional.

Após iniciar o Maltego no Kali Linux, foi criada uma nova análise
(\textit{New}) e adicionada a entidade \textbf{Website}, disponível no grupo
\textit{Infrastructure}. O domínio configurado foi \texttt{ufc.br}, pertencente
à Universidade Federal do Ceará.

Foram executadas diversas \textit{Transforms} para extração de informações
públicas associadas ao domínio:

\begin{itemize}
    \item \textbf{To Domains [within Properties]}: identificação do domínio
      relacionado;
    \item \textbf{To DNSNames [within Properties]}: levantamento de registros
      DNS associados;
    \item \textbf{To Person [PGP]}: identificação de possíveis pessoas
      vinculadas ao domínio;
    \item \textbf{To Email Addresses [PGP]}: coleta de endereços de e-mail
      institucionais expostos;
    \item \textbf{To DNS Name – MX (mail server)}: identificação dos servidores
      de e-mail utilizados pela instituição.
\end{itemize}

Como resultado, foi possível observar a associação do domínio \texttt{ufc.br} a
diversos registros DNS, endereços de e-mail institucionais e servidores de
e-mail (MX), identificando a utilização de infraestrutura de serviços do Google
para correio eletrônico.
}

\newpage
\question[Atividade 1.10]{Conhecendo Ataques Contra Aprendizagem de Máquinas (Adversarial Machine Learning)}
\begin{figure}[H]
  \centering
  \includegraphics[width=1.0\textwidth]{./fig/fig1.10.png}
  \caption{Execução de ataque adversarial no modelo GTSRB e erro de
  classificação após perturbação}
  \label{fig:fig1.10}
\end{figure}
\answer{
Nesta atividade, foi explorado o conceito de \textit{Adversarial Machine
Learning}, que estuda vulnerabilidades em modelos de aprendizado de máquina
quando submetidos a entradas maliciosamente manipuladas.

Foi utilizada a plataforma interativa \texttt{adversarial.js}, acessada via
navegador, permitindo visualizar o comportamento de uma rede neural treinada
para reconhecimento de placas de trânsito (\textbf{GTSRB – German Traffic Sign
Recognition Benchmark}).

Inicialmente, foi selecionado o modelo \textbf{GTSRB (street sign
recognition)}. Na seção \textit{Original Image}, ao executar a opção
\textbf{RUN NEURAL NETWORK}, observou-se que a imagem da placa de trânsito
“STOP” foi corretamente classificada pelo modelo, com alta confiança na
predição.

Em seguida, na seção \textit{Adversarial Image}, foi acionada a opção
\textbf{GENERATE}, que aplicou pequenas perturbações matematicamente calculadas
à imagem original. Essas alterações são praticamente imperceptíveis ao olho
humano, porém são suficientes para modificar os padrões de ativação interna da
rede neural.

Após executar novamente a opção \textbf{RUN NEURAL NETWORK} na imagem
adversarial, verificou-se que o modelo passou a classificar a imagem
incorretamente, demonstrando que o sistema foi enganado com sucesso.

Esse experimento evidencia que modelos de aprendizado de máquina, especialmente
redes neurais profundas, podem ser altamente sensíveis a pequenas perturbações
nos dados de entrada. Em contextos críticos, como veículos autônomos, sistemas
biométricos ou diagnósticos médicos, essas vulnerabilidades podem representar
riscos significativos.
}

