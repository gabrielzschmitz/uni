\section{Atividades}

\question[Atividade 2.6]{Explorando o controle técnico de criptografia de dados
com Ccrypt no Kali Linux}
\begin{figure}[H]
  \centering
  \includegraphics[width=1.0\textwidth]{./fig/fig2.6.png}
  \caption{Arquivo mensagem.txt convertido para mensagem.txt.cpt após
  criptografia com Ccrypt}
  \label{fig:fig2.6}
\end{figure}
\answer{
Nesta atividade, foi explorada a ferramenta \texttt{ccrypt}, utilizada para
implementação de controle técnico de proteção de dados por meio de criptografia
simétrica no Kali Linux.

Inicialmente, foi criado o arquivo \texttt{mensagem.txt} no diretório
\texttt{/home/aluno/Documentos/}, contendo o texto “Hackers do bem!”. Após
confirmar sua criação com o comando \texttt{ls}, foram consultadas as opções da
ferramenta por meio do comando \texttt{ccrypt -h}, a fim de compreender seus
modos de operação e parâmetros disponíveis.

Para aplicar a criptografia, foi executado o comando \texttt{ccrypt -e
mensagem.txt}. O parâmetro \texttt{-e} indica a operação de cifragem
(\textit{encryption}), convertendo automaticamente o arquivo original para
\texttt{mensagem.txt.cpt}. Ao listar o diretório novamente e visualizar o
conteúdo com \texttt{cat mensagem.txt.cpt}, verificou-se que os dados passaram
a apresentar caracteres ilegíveis, evidenciando a aplicação da criptografia.

Em seguida, realizou-se o processo de descriptografia por meio do comando
\texttt{ccrypt -d mensagem.txt.cpt}. O parâmetro \texttt{-d} indica a operação
de decifragem (\textit{decryption}). Após a inserção da mesma chave utilizada
na etapa anterior, o arquivo original \texttt{mensagem.txt} foi restaurado com
seu conteúdo íntegro.

Por fim, o arquivo foi removido com o comando \texttt{rm mensagem.txt},
encerrando a atividade.
}

\newpage
\question[Atividade 2.7]{Explorando os eventos de sistema com o Logcheck no Linux}
\begin{figure}[H]
  \centering
  \includegraphics[width=1.0\textwidth]{./fig/fig2.7.png}
  \caption{Execução do Logcheck em modo workstation exibindo eventos de
  segurança e sistema}
  \label{fig:fig2.7}
\end{figure}
\answer{
Nesta atividade, foi explorada a ferramenta \texttt{Logcheck}, utilizada como
mecanismo de controle detectivo para análise automatizada de logs no Kali
Linux. O objetivo foi identificar eventos relevantes de segurança e sistema a
partir da filtragem das mensagens registradas pelo kernel e serviços ativos.

Inicialmente, foi executado o comando \texttt{sudo -u logcheck logcheck -o -t},
permitindo que o Logcheck operasse no modo online (\texttt{-o}) e exibisse os
relatórios em formato texto simples (\texttt{-t}), sob o contexto do usuário
específico \texttt{logcheck}. A saída apresentou eventos classificados em
\textit{Security Events} e \textit{System Events}.

Entre os eventos de segurança, destacaram-se registros relacionados ao
\texttt{sudo}, indicando abertura e encerramento de sessões privilegiadas para
o usuário root, evidenciando monitoramento de atividades com elevação de
privilégios.

Nos eventos de sistema, foram observadas mensagens do \texttt{dhclient}
relativas a solicitações DHCP na interface \texttt{eth0}, bem como erros do
serviço \texttt{xrdp-chansrv} associados à área de transferência durante
sessões RDP. Esses registros demonstram a capacidade da ferramenta em
identificar tanto eventos operacionais quanto possíveis anomalias.

Posteriormente, foi acessado o arquivo de configuração
\texttt{/etc/logcheck/logcheck.conf} com privilégios administrativos, alterando
o parâmetro \texttt{REPORTLEVEL} para \texttt{"workstation"}, definindo um
nível de filtragem adequado para estações protegidas. Opcionalmente, foi
configurado o parâmetro \texttt{SENDMAILTO} e \texttt{MAILASATTACH} para envio
de relatórios por e-mail.

Após a modificação, o comando foi executado novamente, sendo exibida uma nova
saída contendo registros adicionais de abertura e encerramento de sessões sudo,
bem como novos eventos DHCP e erros do serviço XRDP. Foram capturadas as 20
primeiras linhas conforme solicitado.
}

\newpage
\question[Atividade 2.8]{Explorando o navegador Tor no Kali Linux}
\begin{figure}[H]
  \centering
  \includegraphics[width=1.0\textwidth]{./fig/fig2.8.png}
  \caption{Verificação do endereço IP no Tor Browser demonstrando uso de nó de
  saída da rede Tor}
  \label{fig:fig2.8}
\end{figure}
\answer{
Nesta atividade, foi explorado o navegador \textit{Tor Browser} no Kali Linux,
com o objetivo de compreender seu funcionamento e verificar, na prática, o
mascaramento de endereço IP por meio da rede Tor.

Inicialmente, o navegador foi iniciado com o comando
\texttt{torbrowser-launcher}, responsável por baixar, verificar assinatura
digital e executar a versão mais recente do Tor Browser. Após a abertura, foi
selecionada a opção “Conectar”, permitindo o estabelecimento de uma conexão com
a rede Tor. Em caso de falha inicial, foi utilizada a opção “Try a bridge” para
contornar possíveis restrições de rede.

Com o navegador conectado, foi acessado o site
\texttt{https://duckduckgo.com/}, confirmando a navegação funcional. Em
seguida, foi aberto o site \texttt{https://whatismyipaddress.com/} para
identificação do endereço IP público e da localização aparente.

Os dados apresentados indicaram um endereço IP associado a um \textit{Tor Exit
Node}, com provedor identificado como \textit{Stiftung Erneuerbare Freiheit} e
localização em Frankfurt, Alemanha. Essa informação confirma que o tráfego
estava sendo roteado pela rede Tor, ocultando o endereço IP real da máquina.

Para fins comparativos, o mesmo site foi acessado utilizando o Mozilla Firefox
(fora da rede Tor). Nesse caso, o IP identificado correspondia a um provedor
distinto (Amazon.com Inc.), com localização nos Estados Unidos, evidenciando
navegação convencional sem anonimização.

A diferença entre os endereços IP e localizações demonstrou, de forma prática,
o funcionamento do Tor como mecanismo de anonimização, encaminhando o tráfego
por múltiplos nós distribuídos globalmente.
}

\newpage
\question[Atividade 2.9]{Navegando na DeepWeb pelo navegador Tor no Kali Linux}
\begin{figure}[H]
  \centering
  \includegraphics[width=1.0\textwidth]{./fig/fig2.9.png}
  \caption{Acesso a serviço .onion no Tor Browser demonstrando navegação na
  rede Tor}
  \label{fig:fig2.9}
\end{figure}
\answer{
Nesta atividade, foi realizada a navegação em serviços hospedados na rede Tor,
dando continuidade à configuração realizada na atividade anterior. O objetivo
foi compreender o funcionamento de endereços \texttt{.onion} e observar, na
prática, o acesso a conteúdos disponíveis exclusivamente dentro da rede Tor.

Inicialmente, utilizando o navegador convencional (Firefox), foi acessado um
diretório público que lista endereços da rede Tor. Observou-se que os serviços
da DeepWeb possuem como característica principal a utilização da extensão
\texttt{.onion}, a qual não é resolvida pelo DNS tradicional e exige o uso do
Tor Browser para roteamento adequado.

Em seguida, no Tor Browser já conectado à rede Tor, foi acessado um endereço
\texttt{.onion} de teste. Após alguns instantes de carregamento, o serviço foi
exibido com sucesso, confirmando que o tráfego estava sendo encaminhado por
circuitos da rede Tor até o respectivo serviço oculto. A mensagem apresentada
indicava tratar-se de um serviço na versão v3, modelo atualmente adotado devido
a melhorias criptográficas e de segurança em relação aos antigos serviços v2.

Posteriormente, foram testados outros endereços \texttt{.onion}
disponibilizados na página de referência. Constatou-se que alguns serviços
podem estar temporariamente indisponíveis. Essa indisponibilidade pode ocorrer
por diversos fatores, como instabilidade do serviço, encerramento voluntário,
medidas judiciais ou ataques cibernéticos.

A atividade permitiu observar que:
\begin{itemize}
    \item Serviços \texttt{.onion} só são acessíveis por meio da rede Tor;
    \item O tempo de carregamento tende a ser superior ao da navegação
      convencional, devido ao roteamento por múltiplos nós;
    \item A disponibilidade dos serviços pode variar significativamente;
    \item A navegação deve ser realizada com cautela e responsabilidade.
\end{itemize}

Por fim, todas as janelas dos navegadores (Tor e Firefox ESR) e o Terminal
foram encerradas, concluindo a atividade.
}

\newpage
\question[Atividade 2.10]{Explorando as Políticas Locais e de Conta nas
Configurações de Segurança do Windows Server 2022}
\begin{figure}[H]
  \centering
  \includegraphics[width=1.0\textwidth]{./fig/fig2.10.png}
  \caption{Interface do Local Security Policy (secpol.msc) no Windows Server
  2022}
  \label{fig:fig2.10}
\end{figure}
\answer{
Nesta atividade, foi analisado o console \texttt{Local Security Policy}
(\texttt{secpol.msc}) no Windows Server 2022, com foco na compreensão das
políticas locais e de conta como mecanismos de controle gerencial e técnico de
segurança.

Após o acesso remoto ao servidor via RDP, foi utilizado o campo de busca para
executar o comando \texttt{secpol}, abrindo o console de gerenciamento de
políticas de segurança locais.

\textbf{1. Account Policies}

Dentro de \textit{Account Policies}, foram examinadas duas categorias
principais:

\textit{Password Policy} — Define os critérios aplicáveis às senhas das contas
locais. Entre os parâmetros observados destacam-se:
\begin{itemize}
    \item Histórico de senhas, impedindo reutilização recente;
    \item Idade máxima e mínima da senha, controlando periodicidade de troca;
    \item Comprimento mínimo, fortalecendo a robustez das credenciais;
    \item Requisitos de complexidade, exigindo combinação de caracteres variados.
\end{itemize}

Essas configurações contribuem diretamente para mitigar ataques de força bruta
e reutilização de credenciais.

\textit{Account Lockout Policy} — Responsável por definir regras de bloqueio
após tentativas consecutivas de autenticação malsucedidas. Foram analisados:
\begin{itemize}
    \item Limite de tentativas inválidas;
    \item Tempo de bloqueio da conta;
    \item Intervalo para redefinição do contador.
\end{itemize}

Essas definições atuam como proteção contra ataques de tentativa e erro
(\textit{brute force}).

\textbf{2. Local Policies}

No menu \textit{Local Policies}, foi explorada a seção \textit{Audit Policy},
que permite configurar o registro de eventos de segurança no \textit{Event
Viewer}. Entre as categorias avaliadas:
\begin{itemize}
    \item Auditoria de logon de conta;
    \item Auditoria de gerenciamento de contas;
    \item Auditoria de acesso a objetos;
    \item Auditoria de alterações de política;
    \item Auditoria de uso de privilégios;
    \item Auditoria de eventos do sistema.
\end{itemize}

Essas configurações estabelecem controles detectivos fundamentais para
monitoramento e rastreabilidade de ações realizadas no servidor.

Ainda em \textit{Local Policies}, foi analisado o item \textit{User Rights
Assignment}, que permite atribuir permissões específicas a usuários ou grupos,
como direito de logon local, logon via RDP, desligamento do sistema, entre
outros.

\textbf{3. Security Options}

Na seção \textit{Security Options}, foram observadas configurações críticas
relacionadas à postura de segurança do sistema, incluindo:
\begin{itemize}
    \item Status das contas Administrator e Guest;
    \item Configurações de canal seguro para membros de domínio;
    \item Controle de exibição do último usuário na tela de logon;
    \item Exigência de combinação CTRL+ALT+DEL para autenticação.
\end{itemize}

Essas opções permitem endurecer a superfície de ataque do sistema operacional
por meio de ajustes administrativos.
}
