\section{Atividades}

\question[Atividade 2.1]{Criando um Trojan de Acesso Remoto com o
Social-Engineer Toolkit no Kali Linux}
\begin{figure}[H]
  \centering
  \includegraphics[width=1.0\textwidth]{./fig/fig2.1.png}
  \caption{Geração do payload com SEToolkit e detecção de malware no Kaspersky
  Threat Intelligence Portal}
  \label{fig:fig2.1}
\end{figure}
\answer{
Nesta atividade, foi utilizada a ferramenta \textit{Social-Engineer Toolkit
(SET)}, nativa do Kali Linux, para gerar um \textit{payload} do tipo
\textbf{Windows Meterpreter Reverse\_TCP X64}, simulando a criação de um Trojan
de Acesso Remoto (RAT) para fins exclusivamente acadêmicos.

Após a execução do comando \texttt{setoolkit} com privilégios de superusuário,
foram selecionadas as opções:
\begin{itemize}
    \item \textbf{Social-Engineering Attacks};
    \item \textbf{Create a Payload and Listener};
    \item \textbf{Windows Meterpreter Reverse\_TCP X64}.
\end{itemize}

Foi configurado o endereço IP local (\texttt{LHOST 192.168.98.40}) e a porta de
escuta reversa (\texttt{LPORT 7777}). O SET gerou automaticamente o arquivo
executável \texttt{payload.exe}, armazenado no diretório \texttt{/root/.set/}.

Em seguida, o \textit{listener} foi iniciado via integração com o
\textit{Metasploit Framework}, utilizando o módulo \texttt{multi/handler},
responsável por aguardar conexões reversas provenientes da máquina alvo.

Para fins de análise de segurança, o arquivo \texttt{payload.exe} foi submetido
ao \textit{Kaspersky Threat Intelligence Portal}, onde foi detectado como
malware (nível de detecção 6), confirmando seu comportamento potencialmente
malicioso.
}

\question[Atividade 2.2]{Explorando o Keylogger XSPY no Kali Linux}
\begin{figure}[H]
  \centering
  \includegraphics[width=1.0\textwidth]{./fig/fig2.2.png}
  \caption{Captura de teclas utilizando o XSPY e visualização do arquivo
  teste.log}
  \label{fig:fig2.2}
\end{figure}
\answer{
Nesta atividade, foi explorada a ferramenta \texttt{xspy}, disponível no Kali
Linux, para demonstrar o funcionamento de um \textit{keylogger}, programa capaz
de registrar teclas digitadas no sistema.

Inicialmente, foi acessada a pasta \texttt{/home/aluno/Documentos}, confirmando
que estava vazia. Em seguida, o keylogger foi executado com o redirecionamento
de saída para o arquivo \texttt{teste.log} por meio do comando:

\begin{center}
\texttt{xspy >> teste.log}
\end{center}

Após a inicialização da ferramenta, foi aberto o Editor de Texto e digitada
manualmente a frase \textit{“Hacker do bem!”}. Posteriormente, a execução do
\texttt{xspy} foi encerrada com \texttt{Ctrl + C}.

Ao visualizar o conteúdo do arquivo \texttt{teste.log} utilizando o comando
\texttt{cat teste.log}, foi possível observar o registro das teclas capturadas
durante a execução do keylogger, comprovando o funcionamento da ferramenta.

Por fim, o arquivo foi removido com o comando \texttt{rm teste.log},
restabelecendo o ambiente inicial.
}

\newpage
\question[Atividade 2.3]{Explorando Ransomwares no Kali Linux}
\begin{figure}[H]
  \centering
  \includegraphics[width=1.0\textwidth]{./fig/fig2.3.png}
  \caption{Criação do Ransomware WannaCry e visualização do arquivo executável
  gerado}
  \label{fig:fig2.3}
\end{figure}
\answer{
Nesta atividade, foi explorado um repositório educacional de criação de
Ransomwares no Kali Linux, com foco na compreensão teórica do funcionamento
dessas ameaças. Ressalta-se que todos os procedimentos foram realizados
exclusivamente para fins acadêmicos, sem execução do código malicioso.

Após acessar o diretório \texttt{/curso/Ransomware} e executar o programa
\texttt{python3 Ransomware}, foi possível visualizar o menu interativo com
diferentes famílias de ransomware disponíveis para geração.

Por meio do comando \texttt{show}, foram listadas diversas variantes, incluindo
Cerber, Locky, Petya e WannaCry. Em seguida, foi selecionada a opção \textbf{14
– Ransomware.WannaCry}, resultando na criação de um arquivo compactado
protegido por senha (\texttt{infected}), salvo como \texttt{sdcard} no
diretório raiz.

Ao acessar o arquivo pelo gerenciador de arquivos, verificou-se que seu
conteúdo incluía um executável (\texttt{.exe}) com aproximadamente 3.5MB,
correspondente ao ransomware gerado. O arquivo não foi executado, mantendo-se a
atividade apenas no âmbito de análise estrutural.

Por fim, o arquivo \texttt{sdcard} foi removido do sistema, restaurando o
ambiente inicial.
}

\newpage
\question[Atividade 2.4]{Explorando múltiplos Payloads de Malware no Kali Linux}
\begin{figure}[H]
  \centering
  \includegraphics[width=1.0\textwidth]{./fig/fig2.4.png}
  \caption{Execução do Brutal.sh e geração do payload
  Screen-Rotation-Pranks.ino}
  \label{fig:fig2.4}
\end{figure}
\answer{
Nesta atividade, foi explorado o repositório educacional \texttt{Brutal},
disponível no diretório \texttt{/curso/Brutal}, com o objetivo de compreender a
estrutura e o funcionamento de múltiplos payloads de malware. Todas as ações
foram realizadas exclusivamente para fins acadêmicos, sem execução dos arquivos
gerados em ambiente Windows.

Inicialmente, foi realizado acesso ao Kali Linux via RDP e, no terminal, foi
utilizado o comando \texttt{sudo -i} para obtenção de privilégios de
superusuário. Em seguida, acessou-se o diretório do projeto por meio do comando
\texttt{cd /curso/Brutal}, sendo confirmado seu conteúdo com \texttt{ls}.

O script principal \texttt{Brutal.sh} recebeu permissão de execução com
\texttt{chmod +x Brutal.sh} e foi executado com \texttt{./Brutal.sh}, exibindo
o menu interativo com diversas categorias de payloads.

Foi selecionada a opção \textbf{05 – Payload Prank for attack computer},
acessando o submenu de “brincadeiras” (hoaxes). Em seguida, foi escolhida
novamente a opção \textbf{05 – Screen Rotation Prank}, resultando na criação do
payload \texttt{Screen-Rotation-Pranks.ino}, conforme indicado pela mensagem de
sucesso exibida no terminal (passo solicitado para print).

Posteriormente, utilizando o gerenciador de arquivos Thunar, foi acessado o
diretório \\\texttt{/curso/Brutal/src/prank/}, onde foram visualizados os
arquivos com extensão \texttt{.ino}. Esses arquivos correspondem a scripts que
podem ser utilizados de forma maliciosa quando aplicados a dispositivos HID
programáveis.
}

\newpage
\question[Atividade 2.5]{Explorando como o Windows Defender detecta um
Keylogger como programa malicioso}
\begin{figure}[H]
  \centering
  \includegraphics[width=1.0\textwidth]{./fig/fig2.5.png}
  \caption{Download do StupidKeylogger bloqueado pelo Microsoft Defender
  SmartScreen e visualização do arquivo na pasta Downloads}
  \label{fig:fig2.5}
\end{figure}
\answer{
Nesta atividade, foi analisado o funcionamento dos mecanismos de proteção do
Microsoft Defender SmartScreen ao tentar realizar o download de um Keylogger
disponível publicamente no GitHub.

Após acessar o Windows Server 2022 (cliente) via RDP, foi utilizado o navegador
Microsoft Edge para acessar o repositório \texttt{StupidKeylogger}, no qual é
disponibilizado um exemplo de Keylogger para fins educacionais. A leitura da
documentação permitiu compreender que o programa tem como finalidade capturar e
registrar teclas digitadas no sistema operacional Windows.

Ao tentar baixar o arquivo compactado \texttt{StupidKeylogger-application.zip},
o Microsoft Defender SmartScreen bloqueou automaticamente o download,
classificando-o como inseguro. Mesmo após selecionar as opções “Keep” e
posteriormente “Keep anyway”, o sistema apresentou alertas adicionais indicando
que o aplicativo era considerado potencialmente malicioso.

O arquivo foi então visualizado na pasta \texttt{Downloads} do Explorador de
Arquivos, conforme solicitado no enunciado da atividade. Ressalta-se que o
objetivo do procedimento foi exclusivamente observar o comportamento do
mecanismo de defesa do Windows diante de um software potencialmente malicioso.
}
