\section{Atividades}

\question[Atividade 5.1]{Criando o Active Directory no Windows Server 2022}
\begin{figure}[H]
  \centering
  \includegraphics[width=1.0\textwidth]{./fig/fig5.1.png}
  \caption{Instalação da função Active Directory Domain Services no Windows
  Server 2022}
  \label{fig:fig5.1}
\end{figure}
\answer{
Nesta atividade, foi realizado o processo de criação de um Active Directory no
Windows Server 2022, evidenciando a configuração inicial de um controlador de
domínio em um ambiente corporativo.

Inicialmente, foi estabelecida a conexão remota ao servidor por meio do RDP,
utilizando as credenciais administrativas. Em seguida, foi acessado o
utilitário \textit{Server Manager}, responsável pelo gerenciamento centralizado
das funções e recursos do sistema.

Posteriormente, foi realizada a alteração do nome do servidor para
\texttt{DC01}, com o objetivo de padronizar a nomenclatura do controlador de
domínio. Após a modificação, o sistema foi reiniciado para aplicar as
alterações, sendo validada a mudança ao acessar novamente o \textit{Server
Manager}.

Na sequência, foi iniciado o assistente de instalação de funções e recursos por
meio da opção \textit{Add Roles and Features}. Durante esse processo, foram
selecionadas as funções \textit{Active Directory Domain Services} e \textit{DNS
Server}, essenciais para a implementação de um domínio e resolução de nomes na
rede.

Também foram incluídos automaticamente recursos complementares, como
\textit{Group Policy Management} e \textit{Remote Server Administration Tools},
necessários para o gerenciamento do ambiente.

Por fim, a instalação foi confirmada e executada, sendo possível acompanhar o
progresso até sua conclusão. Este procedimento demonstrou a configuração
inicial de um serviço de diretório, fundamental para o gerenciamento
centralizado de usuários, grupos e políticas de segurança em redes
corporativas.
}

\newpage
\question[Atividade 5.2]{Criando o Domain Controller no Active Directory}
\begin{figure}[H]
  \centering
  \includegraphics[width=1.0\textwidth]{./fig/fig5.2.png}
  \caption{Configuração do Domain Controller e verificação do DNS no Windows
  Server 2022}
  \label{fig:fig5.2}
\end{figure}
\answer{
Nesta atividade, foi realizada a promoção do servidor previamente configurado a
um Controlador de Domínio (\textit{Domain Controller}), consolidando a
implantação do Active Directory no ambiente.

Inicialmente, por meio do \textit{Server Manager}, foi acessado o alerta de
configuração pendente, representado por uma bandeira com indicação de atenção.
A partir dessa notificação, foi iniciada a promoção do servidor utilizando a
opção \textit{Promote this server to a domain controller}.

Durante o assistente de configuração, foi selecionada a opção de criação de uma
nova floresta (\textit{Add a new forest}), sendo definido o domínio raiz como
\texttt{aluno.hacker.com}. Em seguida, foram configuradas as credenciais de
segurança do modo de restauração de serviços de diretório (DSRM), garantindo a
proteção administrativa do ambiente.

Na etapa de validação, foi confirmado que o nome NetBIOS foi automaticamente
definido como \texttt{ALUNO}. Após a verificação dos pré-requisitos, que foram
atendidos com sucesso, a instalação foi iniciada, culminando na reinicialização
automática do servidor para aplicação das configurações.

Após o reinício, foi estabelecida uma nova conexão remota via RDP, sendo
realizado o acesso com a conta administrativa do domínio. Caso solicitado pelo
sistema, a senha foi atualizada conforme as políticas de segurança.

Posteriormente, foi verificado no \textit{Server Manager} que os serviços
\textit{Active Directory Domain Services (AD DS)}, \textit{DNS} e \textit{File
and Storage Services} estavam devidamente instalados e ativos.

Na sequência, foi acessado o gerenciador de DNS, onde foram analisadas as zonas
de pesquisa direta (\textit{Forward Lookup Zones}), confirmando a criação do
domínio \texttt{aluno.hacker.com}.

Em seguida, foi criada uma zona de pesquisa reversa (\textit{Reverse Lookup
Zone}), associada à rede \texttt{192.168.98.0/24}, permitindo a resolução de
nomes a partir de endereços IP. Foi verificada a autoridade da zona (SOA) e a
correta resolução de nomes, evidenciada por indicadores positivos no sistema.

Além disso, foi habilitada a atualização automática do registro PTR para o
servidor, garantindo a correspondência entre nome e endereço IP. Após a
atualização, foi confirmada a presença do registro na zona reversa, validando o
funcionamento do DNS bidirecional.

Ao final, observou-se que o ambiente de domínio estava plenamente funcional,
com serviços de diretório e resolução de nomes corretamente configurados,
caracterizando a conclusão bem-sucedida da promoção do servidor a Domain
Controller.
}

\newpage
\question[Atividade 5.3]{Inserindo um usuário no Domínio do Active Directory}
\begin{figure}[H]
  \centering
  \includegraphics[width=1.0\textwidth]{./fig/fig5.3.png}
  \caption{Criação de usuário no Active Directory e aplicação de políticas de
  grupo}
  \label{fig:fig5.3}
\end{figure}
\answer{
Nesta atividade, foi realizada a criação de um novo usuário no domínio do
Active Directory, bem como a configuração de permissões e políticas de acesso
por meio de Group Policy, garantindo a autenticação e o acesso remoto ao
ambiente.

Inicialmente, por meio do \textit{Server Manager}, foi acessada a ferramenta
\textit{Active Directory Users and Computers}. Dentro do domínio
\texttt{aluno.hacker.com}, na pasta \textit{Users}, foi criado um novo usuário
com os seguintes dados: nome \texttt{Nome1 Sobrenome1} e login \texttt{nome1}.
Durante a criação, foi definida uma senha segura e ajustadas as configurações
para que a senha não expirasse, além de desabilitar a exigência de alteração no
primeiro acesso.

Após a criação, foram configuradas permissões adicionais para o usuário,
incluindo sua inserção no grupo \textit{Remote Desktop Users}, permitindo o
acesso remoto ao servidor via protocolo RDP.

Em seguida, foi realizada a configuração de políticas de grupo utilizando a
ferramenta \textit{Group Policy Management}. Foi criada uma nova Unidade
Organizacional (OU) denominada \texttt{Rede1}, com o objetivo de organizar os
objetos do domínio e aplicar políticas específicas.

Dentro dessa OU, foi criado um novo Objeto de Política de Grupo (GPO) chamado
\texttt{Alunos}, que foi posteriormente editado para definir permissões de
acesso remoto. No editor de políticas, foi configurada a diretiva \textit{Allow
log on through Remote Desktop Services}, adicionando os grupos \textit{Domain
Users} e \textit{Authenticated Users}, garantindo que usuários do domínio
possam realizar login remoto.

Por fim, foi executado o comando \texttt{gpupdate /force} no prompt de comando,
forçando a atualização das políticas de grupo no sistema. A mensagem de sucesso
confirmou que tanto as políticas de computador quanto as de usuário foram
aplicadas corretamente.

Com isso, o ambiente passou a permitir a autenticação de usuários do domínio e
o acesso remoto controlado, evidenciando a correta configuração do Active
Directory e das políticas de segurança associadas.
}

\newpage
\question[Atividade 5.4]{Configurando um cliente para ingressar no domínio do Active Directory}
\begin{figure}[H]
  \centering
  \includegraphics[width=1.0\textwidth]{./fig/fig5.4.png}
  \caption{Configuração do cliente no domínio e autenticação via Active
  Directory}
  \label{fig:fig5.4}
\end{figure}
\answer{
Nesta atividade, foi realizada a configuração de um cliente Windows Server 2022
para ingressar no domínio do Active Directory previamente criado, permitindo a
autenticação centralizada de usuários.

Inicialmente, no servidor, foi utilizado o comando \texttt{ipconfig} para
identificar o endereço IP da máquina, sendo este \texttt{192.168.98.20}, que
será utilizado como servidor DNS para os demais dispositivos da rede.

Em seguida, foi estabelecida uma conexão RDP com a máquina cliente
(\texttt{192.168.98.30}), onde foram realizadas as configurações necessárias.
Primeiramente, foi verificada a conectividade com o servidor por meio do
comando \texttt{ping 192.168.98.20}, confirmando a comunicação entre as
máquinas.

Posteriormente, foi configurado manualmente o servidor DNS do cliente,
definindo o endereço \texttt{192.168.98.20} como servidor DNS preferencial.
Essa etapa é essencial para que o cliente consiga localizar o controlador de
domínio na rede.

Na sequência, foi acessado o \textit{Server Manager} para instalar o recurso
\textit{Remote Assistance}, garantindo suporte remoto ao sistema. Em seguida,
nas configurações avançadas do sistema, o cliente foi associado ao domínio
\texttt{aluno.hacker.com}, utilizando as credenciais do usuário previamente
criado no Active Directory.

Após a autenticação, o sistema confirmou a inclusão no domínio e foi realizada
a reinicialização da máquina para aplicar as alterações.

Após o reinício, foram configuradas as permissões de acesso remoto, habilitando
conexões RDP e adicionando o usuário do domínio \texttt{nome1} à lista de
usuários autorizados para acesso remoto.

Por fim, foi realizada a autenticação no cliente utilizando as credenciais do
domínio (\texttt{ALUNO\textbackslash nome1}), validando que o computador foi
corretamente integrado ao Active Directory e que o usuário possui permissão de
acesso ao sistema.

Este procedimento demonstra o processo de integração de um cliente a um domínio
Active Directory, evidenciando a centralização da autenticação e o controle de
acesso em ambientes corporativos.
}

\newpage
\question[Atividade 5.5]{Configurando política de senhas no GPO do Windows Server 2022}
\begin{figure}[H]
  \centering
  \includegraphics[width=1.0\textwidth]{./fig/fig5.5.png}
  \caption{Aplicação da política de senhas via Group Policy no domínio}
  \label{fig:fig5.5}
\end{figure}
\answer{
Nesta atividade, foi realizada a configuração de uma política de senhas no
Active Directory por meio do Group Policy Object (GPO), com o objetivo de
reforçar a segurança das credenciais utilizadas no domínio.

Inicialmente, foi acessado o \textit{Server Manager} no Windows Server 2022 e,
por meio da opção \textit{Tools}, foi aberta a ferramenta \textit{Group Policy
Management}. Em seguida, navegou-se até o domínio \texttt{aluno.hacker.com},
onde foi criada uma nova política denominada \texttt{Politica de senhas} dentro
da pasta \textit{Group Policy Objects}.

Após a criação, a política foi editada utilizando o \textit{Group Policy
Management Editor}, onde foram configuradas as diretrizes de senha localizadas
em \textit{Computer Configuration > Policies > Windows Settings > Security
Settings > Account Policies > Password Policy}.

Foram definidos os seguintes parâmetros de segurança:

\begin{itemize}
  \item Idade mínima da senha configurada para 120 dias, garantindo a expiração
    periódica das credenciais;
  \item Comprimento mínimo da senha definido em 9 caracteres, aumentando a
    resistência contra ataques de força bruta;
  \item Habilitação da complexidade de senha, exigindo a combinação de
    diferentes tipos de caracteres;
  \item Histórico de senhas configurado para armazenar as últimas 3 senhas,
    prevenindo a reutilização imediata.
\end{itemize}

Após a configuração, a política foi vinculada ao domínio
\texttt{aluno.hacker.com}, garantindo que todos os computadores e usuários do
domínio estejam sujeitos às regras estabelecidas.

Por fim, foi utilizado o comando \texttt{gpupdate /force} no \textit{Command
Prompt} para forçar a atualização das políticas no sistema, confirmando a
aplicação das configurações definidas.

Este procedimento evidencia a utilização de políticas de grupo para o
fortalecimento da segurança em ambientes corporativos, promovendo boas práticas
de gerenciamento de credenciais e mitigando riscos relacionados a senhas fracas
ou reutilizadas.
}
