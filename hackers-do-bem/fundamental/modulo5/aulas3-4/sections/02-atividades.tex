\section{Atividades}

\question[Atividade 5.6]{Implementando o Controle de Acesso Discricionário
(DAC) no Windows Server 2022}
\begin{figure}[H]
  \centering
  \includegraphics[width=1.0\textwidth]{./fig/fig5.6.png}
  \caption{Configuração de permissões de acesso em arquivos no Windows Server
  2022}
  \label{fig:fig5.6}
\end{figure}
\answer{
Nesta atividade, foi realizado o processo de implementação do Controle de
Acesso Discricionário (DAC) no Windows Server 2022, demonstrando a aplicação de
permissões de segurança em arquivos e diretórios.

Inicialmente, foi estabelecida uma conexão remota ao Windows Server 2022
(cliente) por meio do protocolo RDP, utilizando um usuário de domínio com
permissões padrão. Em seguida, foi criado um arquivo de teste por meio do
\textit{Notepad}, denominado \texttt{Arquivo1.txt}, armazenado na pasta
\texttt{Documents} do usuário.

Posteriormente, por meio do \textit{File Explorer}, foram acessadas as
propriedades da pasta \\\texttt{Documents}, especificamente a aba de segurança
(\textit{Security}), onde são exibidos os usuários e grupos com permissões
associadas ao recurso.

Na etapa seguinte, foi realizada a análise das permissões existentes,
verificando que o grupo \textit{Administrators} possui controle total sobre o
diretório. Tentou-se remover esse grupo, porém o sistema impediu a ação,
evidenciando a proteção aplicada às contas administrativas.

Em seguida, foram modificadas as permissões do usuário \textit{Nome1},
aplicando-se a negação (\textit{Deny}) de controle total sobre a pasta. Após a
confirmação das alterações, o acesso ao diretório foi imediatamente
restringido, impedindo a visualização e manipulação dos arquivos.

Ao tentar acessar novamente a pasta \texttt{Documents}, o sistema apresentou
mensagens de negação de acesso, sendo necessário fornecer credenciais
administrativas para prosseguir, o que demonstra a hierarquia de privilégios no
sistema operacional.

Por fim, as permissões de negação foram removidas, restaurando o acesso do
usuário ao diretório. A validação foi realizada ao acessar novamente a pasta,
confirmando que o conteúdo estava disponível.
}

\newpage
\question[Atividade 5.7]{Criando uma Organizational Unit (OU) no Windows Server
2022}
\begin{figure}[H]
  \centering
  \includegraphics[width=1.0\textwidth]{./fig/fig5.7.png}
  \caption{Criação de uma Unidade Organizacional (OU) e de um usuário no Active
  Directory}
  \label{fig:fig5.7}
\end{figure}
\answer{
Nesta atividade, foi realizado o processo de criação de uma Unidade
Organizacional (\textit{Organizational Unit -- OU}) no Windows Server 2022, bem
como a adição de um usuário ao domínio, evidenciando a organização hierárquica
de objetos no Active Directory.

Inicialmente, foi estabelecida a conexão remota ao servidor por meio do
protocolo RDP, utilizando credenciais administrativas. Em seguida, foi acessada
a ferramenta \textit{Active Directory Users and Computers}, responsável pelo
gerenciamento de objetos no diretório.

Posteriormente, no domínio previamente configurado \texttt{aluno.hacker.com},
foi criada uma nova Unidade Organizacional denominada \texttt{TI}, com o
objetivo de estruturar logicamente os recursos e facilitar a aplicação de
políticas e administração de usuários.

Na sequência, foi realizada a criação de um novo usuário dentro da OU
\texttt{TI}, por meio da opção de inclusão de novos objetos do tipo
\textit{User}. Foram definidos os atributos do usuário, incluindo nome,
sobrenome e o \textit{User logon name}, que corresponde ao identificador de
login (também conhecido como \textit{Common Name}).

Em seguida, foi configurada uma senha de acesso para o usuário, sendo
desabilitada a obrigatoriedade de alteração no primeiro logon e ativada a opção
de senha sem expiração, garantindo a continuidade do acesso para fins de
laboratório.

Após a finalização do assistente, foi possível verificar a criação do novo
usuário dentro da OU \texttt{TI}, confirmando o sucesso da operação.
}

\newpage
\question[Atividade 5.8]{Implementando o Controle de Acesso Discricionário
(DAC) no Kali Linux}
\begin{figure}[H]
  \centering
  \includegraphics[width=1.0\textwidth]{./fig/fig5.8.png}
  \caption{Alteração das permissões do arquivo texto.txt utilizando o comando
  chmod no Kali Linux}
  \label{fig:fig5.8}
\end{figure}
\answer{
Nesta atividade, foi realizado o processo de implementação do Controle de
Acesso Discricionário (DAC) no Kali Linux, por meio da manipulação de
permissões de arquivos no sistema de arquivos.

Inicialmente, foi estabelecida a conexão com a máquina virtual Kali Linux via
RDP, utilizando as credenciais fornecidas. Em seguida, foi realizada a elevação
de privilégios com o comando \texttt{sudo -i}, permitindo a execução de
operações administrativas no sistema.

Posteriormente, foi acessado o diretório \texttt{/home/aluno/Documentos}, onde
foi criado um arquivo de texto denominado \texttt{texto.txt} por meio do editor
\texttt{nano}. No arquivo, foi inserido um conteúdo simples para fins de teste,
e em seguida o arquivo foi salvo.

Na sequência, foi utilizado o comando \texttt{ls} para listar os arquivos
presentes no diretório, confirmando a criação do arquivo. Em seguida, foi
executado o comando \texttt{ls -l}, que permite visualizar as permissões
associadas ao arquivo, bem como informações como proprietário, grupo, tamanho e
data de modificação.

Foi observado que, inicialmente, o arquivo possuía permissões
\texttt{-rw-r--r--}, indicando que o proprietário possuía permissões de leitura
e escrita, enquanto o grupo e os demais usuários possuíam apenas permissão de
leitura.

Posteriormente, foi aplicado o comando \texttt{chmod u+rwx texto.txt}, com o
objetivo de conceder ao usuário proprietário permissões completas de leitura,
escrita e execução sobre o arquivo. Esse comando é fundamental no modelo DAC,
pois permite que o proprietário do arquivo determine quem pode acessá-lo e de
que forma.

Após a alteração, foi novamente utilizado o comando \texttt{ls -l} para
verificar as mudanças, sendo possível observar que as permissões do arquivo
foram atualizadas para \texttt{-rwxr--r--}, indicando que o proprietário agora
possui controle total sobre o arquivo.
}

\newpage
\question[Atividade 5.9]{Implementando o Acesso Baseado em Função (RBAC) no
Kali Linux}
\begin{figure}[H]
  \centering
  \includegraphics[width=1.0\textwidth]{./fig/fig5.9.png}
  \caption{Configuração de grupo e permissões do arquivo texto.txt para
  controle de acesso baseado em função no Kali Linux}
  \label{fig:fig5.9}
\end{figure}
\answer{
Nesta atividade, foi realizado o processo de implementação do modelo de Acesso
Baseado em Função (RBAC) no Kali Linux, por meio da utilização de grupos e
permissões de arquivos.

Inicialmente, foi estabelecida a conexão com o sistema Kali Linux via RDP,
seguida da elevação de privilégios com o comando \texttt{sudo -i}, permitindo a
execução de tarefas administrativas.

Em seguida, foi acessado o diretório \texttt{/home/aluno/Documentos}, onde se
encontrava o arquivo \texttt{texto.txt}, previamente criado. Foi realizada a
listagem dos usuários e grupos existentes no sistema por meio dos comandos
\texttt{getent passwd} e \texttt{getent group}, evidenciando a estrutura de
contas e grupos do sistema operacional.

Posteriormente, foi criado um novo grupo denominado \texttt{contabilidade}
utilizando o comando \texttt{groupadd}, representando uma função específica no
modelo RBAC. Após a criação, foi confirmada a presença do grupo na lista de
grupos do sistema.

Na sequência, o usuário \texttt{teste1} foi associado ao grupo
\texttt{contabilidade} por meio do comando \texttt{usermod -aG}, atribuindo a
esse usuário a função correspondente ao grupo criado.

Em seguida, foi alterado o grupo proprietário do arquivo \texttt{texto.txt}
utilizando o comando \texttt{chown :contabilidade texto.txt}, vinculando o
recurso ao grupo responsável pelo acesso.

Posteriormente, foram definidas as permissões do arquivo com o comando
\\\texttt{chmod 770 texto.txt}, concedendo permissões completas de leitura,
gravação e execução ao proprietário e ao grupo, enquanto restringe totalmente o
acesso a outros usuários.

Essa configuração demonstra o modelo RBAC, no qual o acesso aos recursos é
controlado com base em funções representadas por grupos, e não diretamente por
usuários individuais, facilitando a administração de permissões em ambientes
com múltiplos usuários.

Por fim, o arquivo \texttt{texto.txt} foi removido, concluindo a atividade.
Este procedimento evidenciou como o uso de grupos e permissões no Linux pode
ser empregado para implementar controle de acesso baseado em função, promovendo
maior organização e segurança no gerenciamento de recursos.
}

\newpage
\question[Atividade 5.10]{Implementando a Rotação de Senha no Kali Linux}
\begin{figure}[H]
  \centering
  \includegraphics[width=1.0\textwidth]{./fig/fig5.10.png}
  \caption{Edição do arquivo /etc/login.defs para configuração de políticas de
  rotação de senha no Kali Linux}
  \label{fig:fig5.10}
\end{figure}
\answer{
Nesta atividade, foi realizada a implementação de políticas de rotação de
senhas no sistema Kali Linux, com o objetivo de aumentar a segurança das contas
de usuário por meio da definição de regras de validade e troca periódica de
senhas.

Inicialmente, foi estabelecida a conexão com o sistema via RDP, seguida da
abertura do terminal e da elevação de privilégios utilizando o comando
\texttt{sudo -i}, permitindo a execução de tarefas administrativas.

Em seguida, foram verificados os parâmetros atuais da senha do usuário
\texttt{root} por meio do comando \texttt{passwd -S}, observando informações
como status da senha, data da última alteração, período mínimo e máximo de
validade, tempo de aviso antes do vencimento e política de desativação da
conta. Foi possível identificar que a senha possuía um tempo de validade muito
elevado, configurado em \texttt{99999} dias.

Posteriormente, foram analisados os parâmetros da senha do usuário
\texttt{aluno}, utilizando o mesmo comando, verificando que as configurações
também apresentavam valores padrão com validade extensa e pouca restrição para
troca de senha.

Na sequência, foi realizada a edição do arquivo de configuração
\texttt{/etc/login.defs} utilizando o editor \texttt{nano}, com o objetivo de
ajustar as políticas globais de senha do sistema.

Os parâmetros foram modificados de:

\begin{itemize}
  \item \texttt{PASS\_MAX\_DAYS 99999}
  \item \texttt{PASS\_MIN\_DAYS 0}
  \item \texttt{PASS\_WARN\_AGE 7}
\end{itemize}

para:

\begin{itemize}
  \item \texttt{PASS\_MAX\_DAYS 270}
  \item \texttt{PASS\_MIN\_DAYS 90}
  \item \texttt{PASS\_WARN\_AGE 10}
\end{itemize}

Essas alterações estabelecem que a senha passa a ter validade máxima de 270
dias, não podendo ser alterada antes de 90 dias após a última mudança, além
de notificar o usuário com 10 dias de antecedência sobre o vencimento da
senha.

Após realizar as modificações, o arquivo foi salvo e o editor encerrado. Com
isso, as novas políticas passam a ser aplicadas para novos usuários e para
configurações futuras de senha.
}
