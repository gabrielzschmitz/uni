\section{Atividades}

\question[Atividade 3.6]{Explorando ferramentas de diagnóstico de rede no Kali Linux}
\begin{figure}[H]
  \centering
  \includegraphics[width=1.0\textwidth]{./fig/fig3.6.png}
  \caption{Visualização das conexões de rede ativas utilizando o comando
  netstat}
  \label{fig:fig3.6}
\end{figure}
\answer{
Nesta atividade, foram exploradas ferramentas nativas do sistema Linux para
diagnóstico e análise de rede no ambiente Kali Linux, com o objetivo de
compreender o funcionamento das interfaces, conectividade e comunicação entre
hosts em uma rede.

Inicialmente, foi realizado o acesso remoto ao sistema Kali Linux e, em
seguida, obtidos privilégios de superusuário por meio do comando \texttt{sudo
-i}, permitindo a execução de comandos administrativos.

O comando \texttt{ifconfig} foi utilizado para identificar as interfaces de
rede disponíveis no sistema, bem como seus respectivos endereços IP, máscaras
de sub-rede e estatísticas de transmissão. Foram observadas interfaces como
\textit{eth0}, responsável pela comunicação com a rede, \textit{lo} (loopback)
e interfaces virtuais como \textit{docker0}.

Em seguida, foi empregado o comando \texttt{ping} para verificar a
conectividade com um servidor externo, permitindo avaliar a disponibilidade do
destino e medir o tempo de resposta (latência). Essa ferramenta também
possibilitou a resolução de nomes de domínio em endereços IP.

O comando \texttt{traceroute} foi utilizado para mapear o caminho percorrido
pelos pacotes até um destino específico. A análise dos saltos intermediários
(\textit{hops}) permitiu compreender a rota utilizada na comunicação, bem como
identificar possíveis pontos de latência ou filtragem de pacotes.

Posteriormente, foi explorado o comando \texttt{netstat}, que fornece
informações detalhadas sobre o estado das conexões de rede. A opção \texttt{-i}
permitiu visualizar estatísticas das interfaces, como pacotes transmitidos e
recebidos, além de erros e descartes.

Com o uso de \texttt{netstat -rn}, foi possível analisar a tabela de roteamento
do sistema, identificando a rota padrão, gateways e redes diretamente
conectadas, o que é essencial para entender como o tráfego é direcionado.

A opção \texttt{netstat -s} foi utilizada para obter estatísticas detalhadas
dos protocolos de rede, incluindo IP, TCP, UDP e ICMP, permitindo uma visão
mais aprofundada do comportamento da comunicação no sistema.

Por fim, o comando \texttt{netstat -anptu} possibilitou a visualização das
conexões ativas e portas em escuta, associando cada conexão a processos
específicos. Essa análise é fundamental para identificar serviços em execução e
monitorar possíveis atividades suspeitas.
}

\newpage
\question[Atividade 3.7]{Criação e teste de um Honeypot com PentBox no Kali
Linux}
\begin{figure}[H]
  \centering
  \includegraphics[width=1.0\textwidth]{./fig/fig3.7.png}
  \caption{Registro de tentativa de acesso capturada pelo Honeypot no PentBox}
  \label{fig:fig3.7}
\end{figure}
\answer{
Nesta atividade, foi implementado um honeypot no ambiente Kali Linux com o
objetivo de simular um serviço vulnerável e registrar tentativas de acesso,
permitindo a observação de possíveis comportamentos maliciosos em rede.

Inicialmente, foi realizado o acesso remoto ao sistema Kali Linux e obtidos
privilégios de superusuário por meio do comando \texttt{sudo -i}. Em seguida,
foi identificado o endereço IP da máquina utilizando o comando
\texttt{ifconfig}, necessário para a realização dos testes de acesso a partir
de outro host.

Posteriormente, foi acessado o diretório da ferramenta PentBox e executado o
script \texttt{pentbox.rb}. No menu principal, foram selecionadas as opções de
ferramentas de rede e, em seguida, o módulo de honeypot.

Foi escolhida a configuração automática (\textit{Fast Auto Configuration}), que
inicializa rapidamente o serviço honeypot na porta 80, permitindo a simulação
de um servidor web e o monitoramento de conexões recebidas.

Após a ativação do honeypot, foi realizado um teste de acesso a partir de uma
máquina com Windows Server 2022, utilizando um navegador web para acessar o
endereço IP do Kali Linux. Ao tentar estabelecer a conexão, foi exibida uma
mensagem de acesso negado, indicando que a requisição foi interceptada pelo
honeypot.

Retornando ao Kali Linux, foi possível observar no terminal o registro da
tentativa de acesso, incluindo o endereço IP de origem, a porta utilizada e os
detalhes da requisição HTTP, como método, cabeçalhos e agente do usuário. Essas
informações são fundamentais para análise de possíveis atividades suspeitas.
}

\newpage
\question[Atividade 3.8]{Enumeração DNS com a ferramenta host no Kali Linux}
\begin{figure}[H]
  \centering
  \includegraphics[width=1.0\textwidth]{./fig/fig3.8.png}
  \caption{Consulta DNS utilizando a ferramenta host para obtenção de
  registros}
  \label{fig:fig3.8}
\end{figure}
\answer{
Nesta atividade, foi realizada a enumeração de informações DNS utilizando a
ferramenta \texttt{host} no ambiente Kali Linux, com o objetivo de identificar
registros associados a domínios e compreender sua estrutura de resolução de
nomes.

Inicialmente, foi acessado o sistema Kali Linux e obtidos privilégios de
superusuário por meio do comando \texttt{sudo -i}. Em seguida, foi executada
uma consulta completa ao domínio \textit{grancursos.com.br}, permitindo a
visualização de diferentes tipos de registros DNS.

Foram identificados registros do tipo A (endereços IPv4) e AAAA (endereços
IPv6), evidenciando que o domínio utiliza múltiplos endereços IP, o que pode
indicar balanceamento de carga e alta disponibilidade. Observou-se também a
presença de registros MX, responsáveis pela definição dos servidores de e-mail
do domínio, apontando para a infraestrutura do Google Workspace.

Adicionalmente, foi possível observar registros avançados do tipo SVCB/HTTPS,
que fornecem informações sobre protocolos suportados (como HTTP/2 e HTTP/3) e
recursos de segurança, como o uso de Encrypted Client Hello (ECH).

Na sequência, foram realizadas consultas específicas utilizando a opção
\texttt{-t} da ferramenta \texttt{host}. A consulta do tipo NS permitiu
verificar os servidores de nomes associados a diferentes domínios, sendo
possível identificar casos em que não há registros explícitos e outros em que
há utilização de provedores externos de DNS, como serviços de CDN.

Por fim, foi realizada a consulta de registros MX de um domínio, evidenciando a
priorização entre servidores de e-mail e a existência de mecanismos de
redundância, garantindo maior confiabilidade na entrega de mensagens.

A atividade demonstrou, de forma prática, como a enumeração DNS pode fornecer
informações relevantes sobre a infraestrutura de um domínio, sendo uma etapa
fundamental em processos de análise de rede e reconhecimento em segurança da
informação.
}

\newpage
\question[Atividade 3.9]{Enumeração DNS com nslookup no Kali Linux}
\begin{figure}[H]
  \centering
  \includegraphics[width=1.0\textwidth]{./fig/fig3.9.png}
  \caption{Consultas DNS utilizando a ferramenta nslookup}
  \label{fig:fig3.9}
\end{figure}
\answer{
Nesta atividade, foi realizada a enumeração de informações DNS utilizando a
ferramenta \texttt{nslookup} no ambiente Kali Linux, com o objetivo de
identificar registros associados a domínios e compreender sua infraestrutura de
resolução de nomes.

Inicialmente, foi acessado o sistema Kali Linux e obtidos privilégios de
superusuário por meio do comando \texttt{sudo -i}. Em seguida, foi executada
uma consulta ao domínio \textit{grancursosonline.com.br}, permitindo a obtenção
de informações básicas de resolução DNS.

A saída apresentou o servidor DNS utilizado na consulta (192.168.98.2), bem
como a porta padrão do serviço (53). A resposta foi classificada como
\textit{non-authoritative}, indicando que as informações retornadas não foram
fornecidas diretamente por um servidor autoritativo do domínio.

Foram identificados múltiplos endereços IP associados ao domínio, incluindo
endereços IPv4 (registros A) e IPv6 (registros AAAA). A presença de múltiplos
endereços sugere o uso de balanceamento de carga e alta disponibilidade. Além
disso, os blocos de IP indicam a utilização de serviços de CDN e proteção, como
os oferecidos pela Cloudflare.

Na sequência, foi realizada a consulta de servidores de nomes (NS) utilizando o
modo interativo do \texttt{nslookup}, por meio do comando \texttt{set type=ns}.
A saída apresentou os servidores de nomes associados ao domínio, evidenciando o
uso de provedores externos de DNS. Essas informações são essenciais para
compreender como o domínio gerencia sua resolução de nomes.

Posteriormente, foi realizada a consulta de registros MX (Mail Exchange),
utilizando o comando \texttt{set type=mx}. Foram identificados diversos
servidores de e-mail com diferentes prioridades, indicando a existência de
mecanismos de redundância e alta disponibilidade. Observou-se também que os
servidores de e-mail pertencem ao Google Workspace, evidenciando a utilização
de serviços terceirizados para gerenciamento de correio eletrônico.

Por fim, foram consultados os registros TXT do domínio utilizando o comando
\texttt{nslookup -type=txt}. A saída apresentou diversos registros utilizados
para verificação de domínio e configuração de serviços, incluindo verificações
de plataformas externas e políticas de segurança como SPF (Sender Policy
Framework). O SPF define quais servidores estão autorizados a enviar e-mails em
nome do domínio, sendo uma medida importante para prevenção de spoofing.
}

\newpage
\question[Atividade 3.10]{Enumeração DNS com dig no Kali Linux}
\begin{figure}[H]
  \centering
  \includegraphics[width=1.0\textwidth]{./fig/fig3.10.png}
  \caption{Consultas DNS utilizando a ferramenta dig}
  \label{fig:fig3.10}
\end{figure}
\answer{
Nesta atividade, foi realizada a enumeração de informações DNS utilizando a
ferramenta \texttt{dig} (Domain Information Groper) no ambiente Kali Linux, com
o objetivo de analisar registros DNS de um domínio e compreender sua estrutura
de resolução de nomes.

Inicialmente, foi acessado o sistema Kali Linux e obtidos privilégios de
superusuário por meio do comando \texttt{sudo -i}. Em seguida, foi verificada a
sintaxe do comando \texttt{dig}, permitindo compreender suas opções e
parâmetros para consultas DNS específicas.

A primeira consulta foi realizada ao domínio \textit{grancursosonline.com.br},
resultando na obtenção de registros do tipo A (endereços IPv4). A saída do
comando apresentou informações detalhadas do protocolo DNS, incluindo o
cabeçalho da resposta, flags, tempo de consulta, servidor utilizado e seções de
pergunta e resposta. Foram identificados múltiplos endereços IP associados ao
domínio, indicando o uso de balanceamento de carga e alta disponibilidade,
característicos de serviços de CDN.

Na sequência, foi realizada a consulta de servidores de nomes (NS) utilizando o
parâmetro \texttt{-t ns}. A saída apresentou os servidores responsáveis pela
resolução do domínio, evidenciando o uso de infraestrutura externa de DNS.
Esses registros são fundamentais para compreender a delegação do domínio.

Posteriormente, foi realizada a consulta de registros MX (Mail Exchange),
utilizando o parâmetro \texttt{-t mx}. Foram identificados diversos servidores
de e-mail com diferentes prioridades, indicando a existência de redundância no
serviço de correio eletrônico. Observou-se que os servidores pertencem ao
Google Workspace, evidenciando a utilização de serviços terceirizados para
gerenciamento de e-mails.

Em seguida, foram consultados os registros AAAA, responsáveis por mapear
endereços IPv6 do domínio. A presença de múltiplos endereços IPv6 reforça a
adoção de boas práticas de disponibilidade e distribuição de carga, além da
compatibilidade com redes modernas.

Por fim, foi realizada a consulta de registros CNAME. A ausência de respostas
nesta seção indica que o domínio consultado não possui um nome canônico
associado. No entanto, a seção de autoridade (SOA) foi apresentada, contendo
informações sobre o servidor DNS primário, o responsável pelo domínio e
parâmetros de controle de cache e sincronização entre servidores.
}
