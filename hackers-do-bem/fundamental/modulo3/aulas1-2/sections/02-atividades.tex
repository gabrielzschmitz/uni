\section{Atividades}

\question[Atividade 3.1]{Explorando Exploits Conhecidos com Exploit Database no Kali Linux}
\begin{figure}[H]
  \centering
  \includegraphics[width=1.0\textwidth]{./fig/fig3.1.png}
  \caption{Utilização do searchsploit para busca de exploits no Exploit
  Database}
  \label{fig:fig3.1}
\end{figure}
\answer{
Nesta atividade, foi explorada a base de dados do \textit{Exploit Database}
presente no Kali Linux, utilizando a ferramenta \texttt{searchsploit} para
pesquisa de vulnerabilidades conhecidas. O objetivo foi compreender como
identificar exploits disponíveis para diferentes softwares e sistemas, com fins
exclusivamente acadêmicos.

Inicialmente, foi realizado acesso ao ambiente Kali Linux via RDP e, em
seguida, obtidos privilégios de superusuário com o comando \texttt{sudo -i}.
Posteriormente, foi verificado o diretório padrão da base de dados do ExploitDB
em \texttt{/usr/share/exploitdb}, onde estão armazenados exploits, shellcodes e
arquivos auxiliares.

A ferramenta \texttt{searchsploit} foi utilizada para realizar consultas na
base local. Foram observadas suas principais funcionalidades, incluindo:
\begin{itemize}
    \item Busca por termos relacionados a softwares ou vulnerabilidades;
    \item Filtros por título, versão e identificadores CVE;
    \item Exibição de resultados em diferentes formatos, como JSON;
    \item Acesso direto ao caminho dos exploits armazenados localmente.
\end{itemize}

Em seguida, a base de dados foi atualizada por meio do comando
\texttt{searchsploit -u}, garantindo que os resultados refletissem as
vulnerabilidades mais recentes disponíveis.

Foram realizadas buscas práticas utilizando diferentes termos. A pesquisa por
\textit{OpenSSL} retornou múltiplos exploits relacionados a falhas conhecidas,
incluindo vulnerabilidades de negação de serviço, execução remota de código e
problemas de criptografia. Também foram apresentados shellcodes e documentos
técnicos associados.

Na sequência, foi realizada a busca por \textit{OpenSSH}, onde foram
identificados exploits relacionados a enumeração de usuários, execução remota
de comandos e escalonamento de privilégios em diferentes versões do serviço.

Por fim, foi executada uma busca específica para \textit{Windows 10},
utilizando o parâmetro \texttt{-t} para restringir os resultados ao título dos
exploits. Essa consulta demonstrou diversas vulnerabilidades associadas a
aplicações e serviços executados no sistema operacional.

Adicionalmente, foi acessado o site oficial do Exploit Database por meio do
navegador web, permitindo explorar a base de dados online e complementar a
análise realizada localmente.
}

\newpage
\question[Atividade 3.2]{Explorando Varreduras de Rede com Nmap no Kali Linux}
\begin{figure}[H]
  \centering
  \includegraphics[width=1.0\textwidth]{./fig/fig3.2.png}
  \caption{Varredura de rede com Nmap identificando hosts ativos e serviços disponíveis}
  \label{fig:fig3.2}
\end{figure}
\answer{
Nesta atividade, foi utilizada a ferramenta \textit{Nmap} (\textit{Network
Mapper}), amplamente empregada em segurança ofensiva e defensiva para
descoberta de hosts, serviços e análise de redes. O objetivo foi compreender
técnicas básicas de varredura em um ambiente controlado, com fins
exclusivamente acadêmicos.

Inicialmente, foi realizado o acesso ao ambiente Kali Linux via RDP e, em
seguida, obtidos privilégios administrativos utilizando o comando de elevação
de permissões. Com isso, foi possível executar comandos que exigem acesso
privilegiado.

Para identificar as interfaces de rede disponíveis, foi utilizado o comando de
listagem de interfaces, permitindo verificar o endereço IP associado à
interface principal (\texttt{eth0}), pertencente à rede
\texttt{192.168.98.0/24}. Essa informação é essencial para definir o escopo da
varredura.

Na primeira etapa, foi realizada uma varredura de descoberta de hosts
utilizando a opção \texttt{-sn}. Esse tipo de varredura, conhecido como
\textit{ping scan}, tem como finalidade identificar quais dispositivos estão
ativos na rede, sem realizar a análise de portas. O resultado apresentou os
hosts que responderam às requisições, confirmando sua disponibilidade.

Em seguida, foi feita a extração dos endereços IP identificados, utilizando o
encadeamento de comandos com \textit{pipes}. O fluxo de dados foi filtrado para
selecionar apenas os IPs encontrados, os quais foram armazenados em um arquivo
de texto denominado \texttt{ips.txt}. Essa abordagem facilita o
reaproveitamento dos dados em análises posteriores.

Posteriormente, foi executada uma varredura do tipo \textit{TCP ACK Scan},
utilizando a opção \texttt{-sA}. Essa técnica é empregada para identificar a
presença de mecanismos de filtragem, como firewalls, analisando as respostas
aos pacotes enviados. Com isso, é possível classificar as portas como filtradas
ou não filtradas, sem necessariamente determinar se estão abertas.

Na sequência, foi realizada a identificação de portas abertas com a opção
\texttt{--open}. Essa varredura apresenta apenas portas consideradas abertas ou
potencialmente acessíveis. Os resultados indicaram que a maior parte dos hosts
possuía portas filtradas, sugerindo a existência de controles de segurança na
rede. Em um dos hosts, foi observada a porta 53/TCP em estado acessível,
associada ao serviço de resolução de nomes (DNS).

Por fim, foi utilizada a opção \texttt{--packet-trace}, que permite visualizar
detalhadamente os pacotes enviados e recebidos durante a varredura. Essa
funcionalidade possibilita analisar o comportamento da comunicação de rede em
baixo nível, incluindo cabeçalhos e respostas dos hosts. A análise revelou a
existência de serviços ativos, como SSH (porta 22/TCP) e RDP (porta 3389/TCP).

Ao término da atividade, o arquivo gerado foi removido, mantendo o ambiente
organizado.
}

\newpage
\question[Atividade 3.3]{Interceptando tráfego de navegador com o Burp Suite no Kali Linux}

\begin{figure}[H]
  \centering
  \includegraphics[width=1.0\textwidth]{./fig/fig3.3.png}
  \caption{Interceptação de requisições HTTP utilizando o Burp Suite}
  \label{fig:fig3.3}
\end{figure}
\answer{
Nesta atividade, foi realizada a análise do tráfego de navegação web por meio
da ferramenta \textit{Burp Suite}, amplamente utilizada em testes de segurança
de aplicações. O objetivo foi compreender o funcionamento de um proxy
interceptador, permitindo visualizar e manipular as requisições enviadas entre
o navegador e os servidores web, sempre com finalidade exclusivamente
acadêmica.

Inicialmente, o ambiente Kali Linux foi acessado remotamente via RDP utilizando
as credenciais fornecidas. Em seguida, o \textit{Burp Suite} foi iniciado a
partir do terminal com o usuário padrão. Durante a inicialização, foram aceitos
os termos de uso da ferramenta e selecionada a opção de projeto temporário em
memória, utilizando as configurações padrão.

Após a inicialização, foi acessada a aba \textit{Proxy}, onde se verificou que
o interceptador estava inicialmente desativado. Também foi possível observar,
nas configurações do proxy, que o serviço estava em execução na interface local
\texttt{127.0.0.1} na porta \texttt{8080}.

Na sequência, o navegador Mozilla Firefox foi configurado manualmente para
utilizar o proxy local do Burp Suite. Para isso, foram ajustadas as
configurações de rede, definindo o endereço IP \texttt{127.0.0.1} e a porta
\texttt{8080}, além de habilitar o uso do proxy também para conexões HTTPS.

Após a configuração, foi acessado o endereço \texttt{http://burp}, permitindo o
download do certificado da autoridade certificadora do Burp Suite. Esse
certificado foi posteriormente importado no navegador, garantindo que conexões
HTTPS pudessem ser interceptadas sem apresentar erros de segurança.

Com o ambiente devidamente configurado, a interceptação foi ativada na aba
\textit{Proxy}. Em seguida, ao acessar um site HTTPS no navegador, foi possível
observar que a requisição foi interceptada pelo Burp Suite antes de chegar ao
destino. Nesse momento, o navegador permaneceu aguardando, enquanto os detalhes
da requisição, como cabeçalhos HTTP, cookies e informações do cliente, eram
exibidos na ferramenta.

Também foi analisado o histórico de requisições na aba \textit{HTTP history},
onde todas as comunicações realizadas pelo navegador são registradas. Essa
funcionalidade permite acompanhar detalhadamente o fluxo de dados entre cliente
e servidor.

Por fim, a requisição interceptada foi liberada manualmente por meio do botão
\textit{Forward}, permitindo que o navegador carregasse a página solicitada.
Após os testes, o proxy foi desativado nas configurações do navegador e as
aplicações foram encerradas.
}

\newpage
\question[Atividade 3.4]{Escutando requisições com o Netcat no Kali Linux}
\begin{figure}[H]
  \centering
  \includegraphics[width=1.0\textwidth]{./fig/fig3.4.png}
  \caption{Captura de requisição HTTP utilizando o Netcat em modo escuta}
  \label{fig:fig3.4}
\end{figure}
\answer{
Nesta atividade, foi utilizada a ferramenta \texttt{Netcat} (nc) para observar
requisições de rede em um ambiente \textit{Kali Linux}. O objetivo foi
entender, de forma prática, como conexões TCP podem ser recebidas e analisadas
em nível de aplicação, evidenciando o funcionamento básico de comunicações em
rede.

Inicialmente, foi realizado o acesso remoto ao sistema via RDP. Em seguida,
foram obtidos privilégios administrativos por meio do comando \texttt{sudo -i},
permitindo a execução de operações que exigem maior nível de permissão.

Na sequência, foi consultada a ajuda da ferramenta com \texttt{nc -help},
permitindo observar os principais parâmetros disponíveis. Entre eles,
destacam-se as opções para definição de protocolo, porta, modo de escuta e
exibição detalhada das conexões.

Para preparar o ambiente de teste, foi identificado o endereço IP da máquina
utilizando o comando \texttt{ifconfig}. O endereço obtido foi essencial para
permitir que o navegador realizasse a conexão com o serviço em escuta.

Em seguida, o Netcat foi configurado para operar como um servidor, utilizando o
comando \texttt{nc -l -p 5555 -v}. Nesse contexto:
\begin{itemize}
  \item \texttt{-l} habilita o modo de escuta;
  \item \texttt{-p 5555} define a porta de atendimento;
  \item \texttt{-v} ativa a saída detalhada das conexões.
\end{itemize}

Com o serviço em execução, foi aberto o navegador web e realizada uma
requisição HTTP para o endereço correspondente ao IP da máquina na porta
configurada. Essa ação resultou no estabelecimento de uma conexão com o Netcat,
que passou a exibir, em tempo real, os dados recebidos.

Foi possível observar informações relevantes da requisição HTTP, como o método
\texttt{GET}, o cabeçalho \texttt{Host}, o campo \texttt{User-Agent} e outros
parâmetros associados às preferências do cliente. Esses dados evidenciam como o
protocolo HTTP transmite metadados importantes durante a comunicação entre
cliente e servidor.

A análise demonstrou que o Netcat pode ser utilizado como uma ferramenta
simples para inspeção de tráfego, permitindo visualizar requisições em texto
puro sem a necessidade de softwares mais complexos. Esse tipo de abordagem é
útil para fins didáticos, testes de conectividade e compreensão do
funcionamento de protocolos de rede.

Por fim, os aplicativos utilizados foram encerrados, concluindo o experimento.
}

\newpage
\question[Atividade 3.5]{Redirecionando tráfego via portas com o Netcat no Kali Linux}
\begin{figure}[H]
  \centering
  \includegraphics[width=1.0\textwidth]{./fig/fig3.5.png}
  \caption{Comunicação entre portas 80 e 443 utilizando redirecionamento com
  Ncat}
  \label{fig:fig3.5}
\end{figure}
\answer{
Nesta atividade, foi demonstrado o redirecionamento de conexões de rede entre
portas distintas utilizando a ferramenta \texttt{Ncat}, presente no ambiente
\textit{Kali Linux}. O objetivo foi compreender, na prática, como o tráfego
pode ser encaminhado entre serviços, permitindo a comunicação indireta entre
diferentes pontos.

Inicialmente, foi realizado o acesso ao sistema e a elevação de privilégios com
o comando \texttt{sudo -i}, possibilitando a execução de serviços em portas
privilegiadas, como a porta 80.

Em seguida, foi iniciado um processo de escuta na porta 80 com o comando
\texttt{ncat -vl 80 -c 'ncat -l 443'}. Nesse cenário, o Ncat foi configurado
para receber conexões na porta 80 e, ao identificar uma nova conexão, executar
outro processo de escuta na porta 443. Dessa forma, estabeleceu-se um mecanismo
de encaminhamento entre essas duas portas.

O modo verboso permitiu acompanhar o estado da aplicação, indicando que o
serviço estava ativo tanto em IPv4 quanto em IPv6. Após a inicialização, foi
aberto um segundo terminal para realizar uma conexão com o endereço local
(\texttt{127.0.0.1}) na porta 80, simulando um cliente acessando o serviço.

A conexão foi estabelecida com sucesso, sendo registrada no terminal que estava
em escuta, evidenciando a recepção da requisição. Em seguida, foi aberto um
terceiro terminal para conectar-se à porta 443 do mesmo host, criando o outro
extremo da comunicação.

Com ambas as conexões ativas, foi possível validar o redirecionamento de dados.
Mensagens enviadas em um dos terminais foram recebidas no outro, demonstrando
que o tráfego estava sendo efetivamente encaminhado entre as portas 80 e 443.
Esse comportamento evidencia a capacidade do Ncat de atuar como um
intermediário na comunicação, permitindo a troca bidirecional de dados.

A atividade também ilustra conceitos fundamentais de redes, como escuta em
portas, estabelecimento de conexões TCP e encaminhamento de tráfego, aspectos
frequentemente explorados em cenários de testes de segurança e análise de rede.
}
