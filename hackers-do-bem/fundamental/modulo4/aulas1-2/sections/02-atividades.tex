\section{Atividades}

\question[Atividade 4.1]{Modificando os parâmetros de Controle de Autenticação
no Kali Linux}
\begin{figure}[H]
  \centering
  \includegraphics[width=1.0\textwidth]{./fig/fig4.1.png}
  \caption{Alteração da senha do usuário no Kali Linux e autenticação com a
  nova credencial}
  \label{fig:fig4.1}
\end{figure}
\answer{
Nesta atividade, foi realizado o processo de modificação de credenciais de
autenticação no sistema Kali Linux, demonstrando o funcionamento do controle de
acesso baseado em senha em ambientes Linux.

Inicialmente, foi verificado o usuário logado e o diretório de trabalho atual
por meio dos comandos \texttt{whoami} e \texttt{pwd}, confirmando o contexto de
execução do usuário \texttt{aluno} no diretório \texttt{/home/aluno}.

Em seguida, foram analisados os arquivos de configuração \texttt{/etc/passwd} e
\texttt{/etc/group}, responsáveis por armazenar informações sobre usuários e
grupos no sistema. Observou-se que o arquivo \texttt{/etc/passwd} contém dados
estruturados em campos, como nome de usuário, UID, GID, diretório pessoal e
shell padrão, enquanto o arquivo \texttt{/etc/group} define os grupos
existentes e seus respectivos membros.

Posteriormente, foi realizada a alteração da senha do usuário \texttt{aluno}
por meio da elevação de privilégios com o comando \texttt{sudo -i}, seguido do
comando \texttt{passwd aluno}, definindo uma nova senha de autenticação.

Após a alteração, foi efetuado o encerramento da sessão e realizado novo acesso
ao sistema via RDP, utilizando a senha atualizada, validando o sucesso do
processo de modificação de credenciais.

Por fim, a senha foi restaurada ao valor original, garantindo a manutenção do
ambiente conforme o estado inicial. Este procedimento evidenciou o controle de
autenticação baseado em senha no Kali Linux e a importância da gestão segura de
credenciais em sistemas operacionais.
}

\newpage
\question[Atividade 4.2]{Explorando a ferramenta SELinux no Kali Linux}
\begin{figure}[H]
  \centering
  \includegraphics[width=1.0\textwidth]{./fig/fig4.2.png}
  \caption{Verificação do status do SELinux no Kali Linux}
  \label{fig:fig4.2}
\end{figure}
\answer{
Nesta atividade, foi realizada a exploração do SELinux (Security-Enhanced
Linux), um mecanismo de segurança que implementa controle de acesso obrigatório
(MAC) em sistemas Linux, permitindo a definição de políticas de segurança mais
granulares sobre processos, arquivos e recursos do sistema.

Inicialmente, foi efetuada a elevação de privilégios utilizando o comando
\texttt{sudo -i}, possibilitando a execução de tarefas administrativas. Em
seguida, procedeu-se com a ativação do SELinux por meio do comando
\texttt{selinux-activate}, que realizou a geração do arquivo de configuração do
GRUB, responsável pelo carregamento do kernel com suporte às políticas de
segurança do SELinux.

Após a ativação, o sistema foi reiniciado para aplicação das alterações.
Durante o processo de inicialização, ocorre a etapa de \textit{relabeling}, na
qual os arquivos do sistema recebem rótulos de segurança apropriados,
permitindo a aplicação correta das políticas definidas.

Com o sistema em execução, foi verificado o estado do SELinux utilizando o
comando \texttt{sestatus}, confirmando que o mecanismo estava habilitado e
operando em modo \textit{permissive}, no qual as violações de política são
registradas em log, sem bloqueio efetivo das ações.

Em seguida, foram analisadas as configurações de usuários do SELinux por meio
do comando \texttt{semanage user -l}, que apresenta os rótulos de segurança,
funções e níveis associados a cada usuário. Também foram verificadas as
associações de login com o comando \texttt{semanage login -l}, evidenciando o
mapeamento entre usuários do sistema e identidades do SELinux.

Posteriormente, foi examinada a lista de variáveis booleanas do SELinux com
\texttt{semanage boolean -l}, permitindo identificar parâmetros que controlam o
comportamento de diferentes serviços e aplicações. Além disso, foram listadas
as definições de portas com o comando \texttt{semanage port -l}, relacionando
tipos de políticas, protocolos e números de portas.

Por fim, foi realizada a desativação do SELinux por meio da edição do arquivo
de configuração \texttt{/etc/selinux/config}, alterando o modo para
\texttt{disabled}, seguida de reinicialização do sistema. A verificação final
com o comando \texttt{sestatus} confirmou a desativação do mecanismo.

Esta atividade demonstrou o funcionamento do SELinux, suas configurações
principais e a importância do controle de acesso baseado em políticas para o
aumento da segurança em sistemas Linux.
}

\newpage
\question[Atividade 4.3]{Criando um cofre de senhas com o KeePassXC no Kali
Linux}
\begin{figure}[H]
  \centering
  \includegraphics[width=1.0\textwidth]{./fig/fig4.3.png}
  \caption{Criação e gerenciamento de credenciais no KeePassXC}
  \label{fig:fig4.3}
\end{figure}
\answer{
Nesta atividade, foi realizada a criação e manipulação de um cofre de senhas
utilizando o KeePassXC, um gerenciador de senhas de código aberto que permite
armazenar credenciais de forma segura por meio de criptografia avançada.

Inicialmente, o aplicativo foi executado no Kali Linux por meio do comando
\texttt{keepassxc}, utilizando o usuário padrão do sistema, conforme
recomendado. Em seguida, foi criado um novo banco de dados, denominado
\textit{Pessoal}, destinado ao armazenamento de credenciais.

Durante a configuração do banco de dados, foi verificado o uso do algoritmo de
criptografia AES de 256 bits, reconhecido por oferecer alto nível de segurança
na proteção dos dados armazenados. Foi então definida uma senha mestra para o
cofre, responsável por controlar o acesso às informações armazenadas.

Após a criação do banco de dados, o arquivo foi salvo no diretório
\texttt{Documentos} com o nome \texttt{Senhas.kdbx}. Na interface do KeePassXC,
foi adicionada uma nova entrada contendo informações de acesso, como título,
nome de usuário, senha e URL, demonstrando o processo de armazenamento de
credenciais.

Posteriormente, foi realizada a exclusão da entrada criada, evidenciando a
capacidade de gerenciamento dos dados armazenados no cofre. Ao final, o arquivo
do banco de dados foi removido do sistema por meio do comando \texttt{rm},
garantindo a limpeza do ambiente de testes.

Esta atividade demonstrou a utilização do KeePassXC para criação de cofres de
senhas seguros, destacando a importância do uso de gerenciadores de credenciais
para a proteção de informações sensíveis em ambientes digitais.
}

\newpage
\question[Atividade 4.4]{Ataque de Dicionário Off-line contra credenciais no
Kali Linux}
\begin{figure}[H]
  \centering
  \includegraphics[width=1.0\textwidth]{./fig/fig4.4.png}
  \caption{Execução do ataque de dicionário com John the Ripper}
  \label{fig:fig4.4}
\end{figure}
\answer{
Nesta atividade, foi realizada a exploração da ferramenta John the Ripper no
Kali Linux, com o objetivo de demonstrar a execução de um ataque de dicionário
off-line para descoberta de senhas a partir de hashes.

Inicialmente, foi obtido acesso privilegiado ao sistema utilizando o comando
\texttt{sudo -i}. Em seguida, foi realizada a leitura do arquivo
\texttt{/etc/shadow}, que armazena os hashes das senhas dos usuários do
sistema, evidenciando a importância do controle de acesso a este arquivo por
conter informações sensíveis.

Posteriormente, foi criado um novo usuário denominado \textit{teste1}, com a
senha previamente definida, utilizando o comando \texttt{useradd} associado à
geração de hash por meio do \texttt{openssl}. Após a criação, foi possível
visualizar o hash correspondente ao usuário no arquivo \texttt{/etc/shadow}.

O hash do usuário criado foi então copiado e armazenado em um arquivo
\texttt{credencial.txt}, localizado no diretório \texttt{Documentos}, com o
objetivo de utilizá-lo como entrada para o processo de quebra de senha.

Na etapa seguinte, foi executada a ferramenta John the Ripper com o comando \\
\texttt{john -format=crypt credencial.txt}, que realizou um ataque de
dicionário utilizando listas de palavras para tentar descobrir a senha
correspondente ao hash informado.

Ao final da execução, a ferramenta foi capaz de identificar corretamente a
senha associada ao usuário \textit{teste1}, demonstrando a eficácia de ataques
de dicionário contra senhas fracas ou previsíveis.

Por fim, o arquivo \texttt{credencial.txt} foi removido do sistema, finalizando
o experimento. Esta atividade evidenciou a importância da adoção de senhas
fortes e do uso de boas práticas de segurança para evitar a exposição de
credenciais a ataques de quebra de senha.
}

\newpage
\question[Atividade 4.5]{Gerenciando credenciais no Windows Server 2022}
\begin{figure}[H]
  \centering
  \includegraphics[width=1.0\textwidth]{./fig/fig4.5.png}
  \caption{Backup das credenciais realizado no Credential Manager}
  \label{fig:fig4.5}
\end{figure}
\answer{
Nesta atividade, foi explorada a ferramenta Credential Manager do Windows
Server 2022, utilizada para o gerenciamento seguro de credenciais armazenadas
no sistema operacional.

Inicialmente, o acesso ao sistema foi realizado via RDP com as credenciais
fornecidas. Em seguida, a ferramenta Credential Manager foi acessada por meio
da barra de pesquisa do sistema.

Dentro da interface da ferramenta, foram analisadas as seções de credenciais
disponíveis. Na aba \textit{Web Credentials}, foi possível visualizar as
credenciais relacionadas a acessos em páginas web, que podem incluir usuários e
senhas salvos no sistema. Já na aba \textit{Windows Credentials}, foram
observadas diferentes categorias de credenciais, como credenciais do Windows,
baseadas em certificado e genéricas, utilizadas para autenticação em serviços,
aplicações e redes.

Posteriormente, foi realizado o processo de backup das credenciais armazenadas.
Para isso, utilizou-se a opção \textit{Back up Credentials}, definindo o
diretório Desktop como local de destino e atribuindo o nome \textit{backup} ao
arquivo gerado.

Durante o procedimento, foi necessário confirmar a operação utilizando a
sequência de segurança \textit{Ctrl + Alt + Delete}, além de definir uma senha
de proteção para o arquivo de backup. Após a confirmação, o sistema indicou que
o backup foi concluído com sucesso.

Ao final, verificou-se a criação do arquivo \texttt{backup.cdr} na área de
trabalho, evidenciando a conclusão do processo. Por questões de segurança, o
arquivo foi removido do sistema, finalizando a atividade.
}
