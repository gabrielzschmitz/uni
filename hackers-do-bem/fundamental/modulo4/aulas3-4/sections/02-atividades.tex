\section{Atividades}

\question[Atividade 4.6]{Implementando um servidor RADIUS com FreeRADIUS no
Kali Linux}
\begin{figure}[H]
  \centering
  \includegraphics[width=1.0\textwidth]{./fig/fig4.6.png}
  \caption{Servidor FreeRADIUS em execução e teste de autenticação utilizando o
  comando radtest}
  \label{fig:fig4.6}
\end{figure}
\answer{
Nesta atividade, foi realizada a implementação de um servidor de autenticação
centralizada utilizando a ferramenta \texttt{FreeRADIUS} no sistema
\textit{Kali Linux}, com o objetivo de explorar o funcionamento do protocolo
RADIUS (\textit{Remote Authentication Dial-In User Service}).

Inicialmente, foi acessado o ambiente via RDP e obtidos privilégios de
superusuário por meio do comando \texttt{sudo -i}. Em seguida, foi realizada a
verificação das interfaces de rede com o comando \texttt{ifconfig}, destacando
a interface de loopback \texttt{127.0.0.1}, utilizada para testes locais de
autenticação.

Posteriormente, foram analisados os arquivos de configuração do serviço no
diretório \\\texttt{/etc/freeradius/3.0/}, bem como o arquivo
\texttt{clients.conf}, onde foi identificado o cliente padrão
\texttt{localhost} com o segredo compartilhado \texttt{testing123}, utilizado
para testes.

Na sequência, foi criado um usuário para autenticação no arquivo
\texttt{users}, definindo o usuário \texttt{usuario1} com a senha
\texttt{rnpesr}. Após a configuração, o servidor foi iniciado em modo de
depuração com o comando \texttt{freeradius -X}, permitindo acompanhar os logs
em tempo real.

Para validar o funcionamento do servidor, foi utilizado o comando
\texttt{radtest usuario1 rnpesr 127.0.0.1 0 testing123}, que realiza uma
requisição de autenticação ao servidor RADIUS. O retorno \textit{Access-Accept}
confirmou que a autenticação foi realizada com sucesso, indicando que o
servidor estava corretamente configurado.

Em seguida, foi realizado um teste com credenciais inválidas, utilizando um
usuário inexistente. Nesse caso, o servidor retornou a resposta
\textit{Access-Reject}, evidenciando a negação de acesso. A análise dos logs
mostrou que a falha ocorreu devido à ausência de uma senha válida cadastrada,
impedindo a definição do método de autenticação.

Dessa forma, a atividade demonstrou na prática o funcionamento do processo de
autenticação, autorização e controle de acesso utilizando o protocolo RADIUS,
bem como a importância da correta configuração de usuários e clientes para o
funcionamento seguro do serviço.
}

\newpage
\question[Atividade 4.7]{Explorando o Google Authenticator no Kali Linux}
\begin{figure}[H]
  \centering
  \includegraphics[width=1.0\textwidth]{./fig/fig4.7.png}
  \caption{Configuração do TOTP com Google Authenticator no Kali Linux por meio
  de QR Code}
  \label{fig:fig4.7}
\end{figure}
\answer{
Nesta atividade, foi realizada a integração do Google Authenticator ao Kali
Linux para implementação de autenticação em dois fatores (2FA) utilizando TOTP
(\textit{Time-based One-Time Password}), fortalecendo o controle de acesso ao
sistema.

Inicialmente, foi acessado o ambiente Kali Linux via RDP e obtidos privilégios
de superusuário com o comando \texttt{sudo -i}. Em seguida, configurou-se
corretamente o fuso horário do sistema com o comando \texttt{timedatectl
set-timezone America/Sao\_Paulo}, garantindo a sincronização de tempo
necessária para o funcionamento adequado dos códigos TOTP.

Posteriormente, foi executado o comando \texttt{google-authenticator},
iniciando o processo de configuração do mecanismo de autenticação. Foi
selecionada a opção de utilização de tokens baseados em tempo (TOTP), sendo
então gerado um QR Code diretamente no terminal, juntamente com uma chave
secreta única.

Utilizando um dispositivo móvel, o QR Code foi escaneado por meio do aplicativo
Google Authenticator, permitindo a geração de códigos temporários de seis
dígitos, atualizados a cada 30 segundos. Em seguida, foi inserido no terminal o
código exibido no aplicativo, confirmando a correta sincronização entre o
servidor e o dispositivo.

Durante a configuração, foram habilitadas medidas adicionais de segurança, como
a atualização do arquivo de configuração \texttt{/root/.google\_authenticator},
a proibição de reutilização de tokens, a ampliação da janela de tolerância de
tempo para compensar possíveis diferenças de sincronização e a limitação de
tentativas de autenticação, reduzindo a exposição a ataques de força bruta.

Dessa forma, a atividade demonstrou a implementação prática de autenticação
multifator no Kali Linux, adicionando uma camada extra de segurança baseada em
algo que o usuário possui (o dispositivo móvel), além da senha tradicional,
elevando significativamente a proteção do sistema.
}

\newpage
\question[Atividade 4.8]{Incorporando o Google Authenticator ao Mozilla
Firefox}
\begin{figure}[H]
  \centering
  \includegraphics[width=1.0\textwidth]{./fig/fig4.8.png}
  \caption{Geração de códigos TOTP no Mozilla Firefox utilizando extensão
  Authenticator}
  \label{fig:fig4.8}
\end{figure}
\answer{
Nesta atividade, foi realizada a integração do mecanismo de autenticação TOTP
ao navegador Mozilla Firefox por meio de uma extensão, permitindo a geração de
códigos de autenticação sem a necessidade de um dispositivo móvel.

Inicialmente, no computador pessoal, foi acessado o site de complementos do
Mozilla Firefox e instalada a extensão “Authenticator”. Após a instalação, o
ícone da extensão passou a ficar disponível na barra de navegação do navegador,
possibilitando o gerenciamento de tokens de autenticação diretamente pelo
browser.

Em seguida, no ambiente Kali Linux, foi obtido acesso como superusuário
utilizando o comando \texttt{sudo -i}. Posteriormente, foi executado o comando
\texttt{google-authenticator} para iniciar a configuração do TOTP. Ao
selecionar a opção de tokens baseados em tempo, foi gerada uma chave secreta
única associada ao usuário.

A chave secreta apresentada no terminal foi copiada para o computador pessoal e
inserida manualmente na extensão Authenticator do Firefox. Para isso, foi
utilizado o recurso de adição manual, preenchendo os campos de emissor e
segredo com as informações correspondentes.

Após a configuração, a extensão passou a gerar códigos temporários
sincronizados com o sistema do Kali Linux, atualizados periodicamente. Dessa
forma, foi possível reproduzir o funcionamento do Google Authenticator
diretamente no navegador, sem a necessidade de um smartphone.

Dessa maneira, a atividade demonstrou a utilização de ferramentas alternativas
para implementação de autenticação em dois fatores, evidenciando a
flexibilidade do uso de TOTP e a possibilidade de integração com diferentes
plataformas, mantendo o mesmo nível de segurança no processo de autenticação.
}

\newpage
\question[Atividade 4.9]{Verificando a presença de TPM e USB no Kali Linux}
\begin{figure}[H]
  \centering
  \includegraphics[width=1.0\textwidth]{./fig/fig4.9.png}
  \caption{Visualização das portas USB no Kali Linux utilizando o USB Viewer}
  \label{fig:fig4.9}
\end{figure}
\answer{
Nesta atividade, foi realizada a verificação da presença do módulo TPM (Trusted
Platform Module) e das portas USB no sistema Kali Linux, com o objetivo de
identificar possíveis dispositivos de segurança e meios de entrada de tokens
físicos.

Inicialmente, foi obtido acesso privilegiado no sistema por meio do comando
\texttt{sudo -i}. Em seguida, utilizou-se o comando \texttt{journalctl -k
--grep=tpm} para buscar registros do kernel relacionados ao TPM. A saída
indicou a mensagem \textit{``No TPM chip found, activating TPM-bypass!''},
evidenciando que a máquina virtual não possui um módulo TPM físico. Dessa
forma, o sistema ativa um modo de bypass, permitindo o funcionamento sem os
recursos de segurança fornecidos por esse hardware.

Também foi possível observar mensagens do \textit{systemd}, indicando os
componentes de segurança e funcionalidades habilitadas no sistema, como PAM,
auditoria e mecanismos de integridade, demonstrando que o sistema continua
operando mesmo na ausência do TPM.

Na sequência, foi realizada a análise das portas USB utilizando o comando
\texttt{usbview}. Ao executar a ferramenta, foi exibida a interface gráfica do
programa “USB Viewer”, responsável por listar os dispositivos USB conectados ao
sistema. No entanto, não foram identificados dispositivos, o que é esperado em
um ambiente virtualizado, onde não há conexão direta com hardware físico.

Dessa forma, a atividade demonstrou como verificar a presença de recursos de
segurança em nível de hardware e identificar dispositivos de entrada,
destacando as limitações inerentes a ambientes virtuais em relação ao acesso a
componentes físicos como TPM e dispositivos USB.
}

\newpage
\question[Atividade 4.10]{Gerenciando TOTPs com o OTPClient no Kali Linux}
\begin{figure}[H]
  \centering
  \includegraphics[width=1.0\textwidth]{./fig/fig4.10.png}
  \caption{Gerenciamento de tokens TOTP no OTPClient com base na chave gerada
  pelo Google Authenticator}
  \label{fig:fig4.10}
\end{figure}
\answer{
Nesta atividade, foi explorada a utilização do OTPClient para gerenciamento de
tokens TOTP (Time-based One-Time Password) no Kali Linux, dando continuidade à
implementação de autenticação em dois fatores (2FA).

Inicialmente, o OTPClient foi iniciado por meio do comando \texttt{otpclient}.
Durante a execução, foi exibido um aviso relacionado ao limite de memória
segura (\textit{memlock}), o qual não impede o funcionamento da ferramenta e
foi confirmado para prosseguir. Em seguida, foi criada uma nova base de dados
criptografada, armazenada no diretório \texttt{/home/aluno/Documentos/} com o
nome \texttt{NewDatabase.enc}, protegida por uma senha definida durante o
processo.

Posteriormente, em um segundo terminal, foi executado o comando
\texttt{google-authenticator} para gerar uma chave secreta associada a um token
TOTP. Ao selecionar a opção de geração baseada em tempo, foi apresentada uma
chave secreta, utilizada para configuração manual no OTPClient.

De volta ao OTPClient, foi adicionada uma nova entrada por meio da opção
“Manually”, na qual foram definidos os campos de identificação do token, como
\textit{Account} e \textit{Issuer}. A chave secreta gerada anteriormente foi
inserida no campo \textit{Secret}, permitindo a geração de códigos TOTP válidos
diretamente na aplicação.

Ao selecionar o token criado, foi possível visualizar seus parâmetros,
incluindo tipo (TOTP), conta, emissor, valor do código gerado e tempo de
validade, confirmando o correto funcionamento do mecanismo de autenticação
baseado em tempo.

Por fim, a base de dados criada foi removida do sistema, juntamente com os
arquivos auxiliares gerados durante o processo, utilizando o comando \texttt{rm
*}, garantindo a limpeza do ambiente.

Dessa forma, a atividade demonstrou como gerenciar tokens de autenticação de
dois fatores localmente, sem a necessidade de dispositivos externos, reforçando
práticas de segurança no acesso a sistemas.
}
