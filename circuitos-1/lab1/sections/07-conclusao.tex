\section{Conclusão}

Podemos concluir que uma simulação computacional do circuito é útil para análise teórico mas ignora imperfeições do mundo real, enquanto o experimento em laboratório revela efeitos práticos (tolerâncias, ruído, etc), mas exige cuidado com medições e calibração, mas que ambos são ferramentas confiáveis de desenvolvimento do estudo de circuitos elétricos e que o resultado de ambos coincidiram, com certa taxa de variação, com os resultados teóricos esperados. Conferimos que seja adequado usar a simulação para validar o conceito, mas sempre ajustar o projeto com testes reais, considerando tolerâncias e efeitos não ideais.