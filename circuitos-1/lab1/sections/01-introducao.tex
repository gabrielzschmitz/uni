\section{Introdução}

A presente atividade tem como objetivo verificar os conceitos de análise de
circuitos elétricos em corrente contínua (DC), com ênfase na aplicação das leis
fundamentais dos circuitos, como a \textit{Lei de Ohm} e as \textit{Leis de
Kirchhoff}. O foco será avaliar como esses conceitos se manifestam na prática,
por meio de experimentos que envolvem a medição de grandezas elétricas em
circuitos reais.

A fundamentação teórica que sustenta esta atividade pode ser encontrada nas obras: \citetitle{alexander2008} \cite{alexander2008}, \citetitle{hayt2008} \cite{hayt2008},
\citetitle{boylestad2012} \cite{boylestad2012}, \citetitle{svoboda2016} \cite{svoboda2016},
\citetitle{nilsson2016} \cite{nilsson2016} e \citetitle{irwin2000} \cite{irwin2000},
que abordam em profundidade os princípios da análise de circuitos elétricos.

Além disso, a atividade busca familiarizar os futuros engenheiros com o uso de
equipamentos de medição, essenciais para a análise e avaliação da resposta de
circuitos. Através deste laboratório, espera-se que os alunos adquiram
experiência prática, desenvolvendo habilidades cruciais para a interpretação e
solução de problemas relacionados à análise de circuitos DC.
