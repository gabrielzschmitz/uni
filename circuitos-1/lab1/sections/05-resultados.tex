\section{Resultados}\label{sec:resultados}

Nesta seção, são apresentados os resultados obtidos nas etapas de análise
teórica, simulação computacional e medição prática do circuito proposto. A
comparação entre os três métodos permite avaliar a consistência dos conceitos
aplicados, a precisão dos modelos simulados e o comportamento real dos
componentes em laboratório. As tabelas a seguir organizam os dados de tensão,
corrente e potência para cada elemento do circuito, destacando eventuais
discrepâncias e reforçando a importância da análise crítica dos resultados
experimentais.

\begin{table}[H]
\centering
\begin{tabularx}{0.9\textwidth}{l *{7}{>{\centering\arraybackslash}X}}
\textbf{Elemento} & \multicolumn{2}{c}{\textbf{Resultado Teórico}} & \multicolumn{2}{c}{\textbf{Resultado Simulado}} & \multicolumn{2}{c}{\textbf{Resultado Real}} \\
 & Tensão (V) & Corrente (mA) & Tensão (V) & Corrente (mA) & Tensão (V) & Corrente (mA) \\
\hline
Fonte & 12.00 & 2.69  & 12.00  & 2.695  & 12.01 & 2.68  \\
R1    & 4.70  & 2.1   & 4.717  & 2.144  & 4.73  & 2.1 \\
R2    & 5.49  & 0.54  & 5.510  & 0.551  & 5.54  & 0.5 \\
R3    & 0.79  & 1.41  & 0.793  & 1.416  & 0.79  & 1.3 \\
R4    & 7.28  & 0.72  & 7.283  & 0.728  & 7.27  & 0.6 \\
R5    & 6.51  & 1.97  & 6.490  & 1.967  & 6.46  & 1.9 \\
\end{tabularx}
\caption{Comparação dos resultados teóricos, simulados e reais dos elementos do circuito}
\end{table}

\begin{table}[H]
\centering
\begin{tabularx}{0.7\textwidth}{lXXX}
\textbf{Elemento} & \textbf{Potência Teórica (mW)} & \textbf{Potência Simulada (mW)} & \textbf{Potência Real (mW)} \\
\hline
Fonte & 32.28 & 32.291 & 32.187 \\
R1    & 10.06 & 10.113 & 9.933  \\
R2    & 3.01  & 3.036  & 2.77   \\
R3    & 1.11  & 1.122  & 1.027  \\
R4    & 5.29  & 5.304  & 4.362  \\
R5    & 12.82 & 12.765 & 12.274 \\
\end{tabularx}
\caption{Comparação de potência dissipada nos resistores do circuito}
\end{table}