\newpage
\section{Análise dos Resultados}

Após realizar análises nas três modalidades, teórica, simulado e experimental, verificamos que os valores calculados entre si são bem próximos ou idênticos, o que serve como um grande indicativo que os nossos resultados são coerentes, contudo ao investigar as discrepâncias entre os valores obtidos, é possível observar alguns motivos pelo ocorrido. 

Primeiramente, como já apontado na secção do experimento laboratorial, os resistores tem um nível de tolerância pelo qual pode variar sua resistência, isso por si só já implicaria numa diferença nas tensões e correntes do circuito dos experimento real, comparado com os cálculos teóricos e a simulação. Investigando mais a frente, podemos conferir que tanto os cálculos teóricos realizados quanto o modelo de simulação utilizado no \(MATLAB\) foram feitos sobre a consideração de que o circuito utilizado como objeto de estudo era ideal ou perfeito, isto é, seus resistores têm valor exato e sem tolerância, fontes de tensão/corrente são perfeitas com nenhuma resistência internas e fios e conexões têm resistência e indutância desprezíveis, o que não acontece num experimento real, onde todas essas limitações físicas inevitavelmente afetam a medição das grandezas de um circuito elétrico.

Além das diferenças derivadas de natureza física do circuito elétrico, existe também um ponto a ser levantando que é o dos métodos numéricos utilizados, que pode ser observado não só em contrasto com o experimento real mas também o teórico e simulado, pois ao realizar a análise nodal foi feito diversas aproximações e trucamentos para viabilizar e facilitar os cálculos, enquanto o modelo matemático do \(MATLAB\) também, apesar de ser muito mais preciso, utiliza métodos como erros de discretização e arredondamentos numéricos como limitações da precisão de ponto flutuante, que pode alterar com certo nível de precisão, os valores esperados de até mesmo um sistema ideal. 

E por fim, temos também as varáveis do ambiente onde se foi executado o experimento real, como instrumentos de medição (multímetros, osciloscópios) que têm precisão limitada, ruídos eletromagnéticos que podem interferir nas medidas e variações térmicas, pois resistores mudam de valor de acordo com a temperatura.