\section{Introdução}

Esta atividade tem como objetivo consolidar os conceitos fundamentais da análise
de circuitos elétricos, com foco na aplicação de Amplificadores Operacionais
(AmpOps). Esses dispositivos são essenciais em diversas configurações, como
amplificadores inversores, não-inversores, somadores e integradores. Busca-se,
assim, compreender tanto o funcionamento teórico quanto o comportamento prático
dos AmpOps, integrando análise analítica, implementação e medições em
laboratório.


A fundamentação teórica que sustenta esta atividade pode ser encontrada nas
obras: \citetitle{alexander2008} \cite{alexander2008}, \citetitle{hayt2008}
\cite{hayt2008}, \citetitle{boylestad2012} \cite{boylestad2012},
\citetitle{svoboda2016} \cite{svoboda2016}, \citetitle{nilsson2016}
\cite{nilsson2016} e \citetitle{irwin2000} \cite{irwin2000}, que abordam em
profundidade os princípios da análise de circuitos elétricos.

Além do embasamento teórico, a atividade visa proporcionar aos futuros
engenheiros uma vivência prática no uso de instrumentos de medição, essenciais
para a caracterização e avaliação do comportamento de circuitos eletrônicos. Por
meio da realização deste experimento em laboratório, espera-se que os alunos
desenvolvam competências técnicas importantes, como a interpretação de
resultados experimentais e a aplicação eficaz dos métodos de análise de
circuitos em corrente contínua (DC).
