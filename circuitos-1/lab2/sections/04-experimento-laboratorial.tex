\section{Experimento Laboratorial}

Nesta etapa, o circuito proposto foi montado fisicamente em bancada utilizando
componentes reais disponíveis no laboratório.

\begin{figure}[H]
  \centering
  \includegraphics[width=0.5\linewidth]{fig/lab2realcircuit1.jpeg}
  \caption{Circuito 1 montado em laboratório}
  \label{fig:real-circuit}
\end{figure}
\begin{figure}[H]
  \centering
  \includegraphics[width=0.5\linewidth]{fig/lab2realcircuit2.jpeg}
  \caption{Circuito 2 montado em laboratório}
  \label{fig:real-circuit}
\end{figure}

Na realização do experimento prático, foram utilizados componentes reais cujos valores apresentam pequenas variações em relação aos valores nominais especificados no projeto teórico.

Os resistores apresentaram tolerâncias típicas de fabricação, o que explica as pequenas diferenças observadas. A fonte de tensão também apresentou uma leve variação, fornecendo \(5,08V\) em vez dos \(5,00V\) ideais.

A tabela a seguir resume a comparação entre os valores teóricos e os valores reais medidos dos componentes:

\begin{itemize}
    \item \(R_3\): 1k\(\Omega\) (teórico) → 0,97k\(\Omega\) (real)
    \item \(R_4\): 1k\(\Omega\) (teórico) → 0,98k\(\Omega\) (real)
    \item \(R_1\): 4,7k\(\Omega\) (teórico) → 4,61k\(\Omega\) (real)
    \item \(R_4\): 4,7k\(\Omega\) (teórico) → 4,61k\(\Omega\) (real)
    \item \(R_2\): 10k\(\Omega\) (teórico) → 9,78k\(\Omega\) (real)
    \item \(R_4\): 10k\(\Omega\) (teórico) → 9,78k\(\Omega\) (real)
    \item \(R_L\): 100\(\Omega\) (teórico) → 115\(\Omega\) (real)
    \item Fonte de tensão: 5,00V (teórico) → 5,08V (real)
\end{itemize}

Essas discrepâncias ocorrem principalmente devido às tolerâncias de fabricação dos resistores, que geralmente variam entre 1\% e 5\%, dependendo do componente. Além disso, o resistor de carga \(R_L\) apresentou uma variação significativa, passando de 100\(\Omega\) para 115\(\Omega\), o que impacta diretamente os valores das correntes medidas.

A variação da tensão da fonte também contribui para pequenas diferenças nos resultados práticos em relação aos teóricos e simulados. Apesar dessas variações, os resultados obtidos são considerados satisfatórios, uma vez que estão dentro dos limites aceitáveis para um experimento prático de laboratório.
