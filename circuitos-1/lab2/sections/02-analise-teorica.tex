\section{Análise Teórica do Circuito 1}

Para a realização da análise teórica do circuito 1, foi adotada a metodologia de análise nodal. 

No circuito, a tensão \(V_A\) representa a tensão no nó entre o resistor \(R_4\) e a entrada inversora do amplificador operacional, sendo também a tensão aplicada sobre o resistor \(R_4\).

A tensão \(V_3\) corresponde à tensão no nó de saída do amplificador, que está conectada diretamente ao resistor de carga \(R_L\), portanto, é a tensão aplicada sobre esse resistor.

A corrente \(I_L\) representa a corrente que circula através do resistor de carga \(R_L\), sendo calculada pela relação:
\[
I_L = \frac{V_3}{R_L}
\]
onde \(R_L\) possui valor fixo de 100\(\Omega\).

\begin{figure}[H]
  \centering
  \includegraphics[width=0.5\linewidth]{fig/lab2circuit1.png}
  \caption{Circuito 1 para análise em laboratório}
  \label{fig:circuit1}
\end{figure}

O circuito 1 foi analisado para três valores de \(R_4\). A seguir são apresentados os cálculos completos via análise nodal.

\subsection{Para \(R_4 = 4,7k\Omega\)}

\subsubsection*{Nó \(v_a\):}
Aplicando a Lei de Kirchhoff dos nós:
\[
\frac{v_a}{4,7k} + \frac{v_a - 5}{1k} = 0
\]
Multiplicando a equação por 4700 para eliminar os denominadores:
\[
1000v_a + 4700(v_a - 5) = 0
\]
\[
1000v_a + 4700v_a - 23500 = 0
\]
\[
5700v_a = 23500
\]
\[
v_a = \frac{23500}{5700} = 4,1226V
\]

\subsubsection*{Nó \(v_3\):}
\[
\frac{4,1226 - v_3}{4,7k} + \frac{v_3}{10k} = 0
\]
Multiplicando por 47000 (MMC):
\[
10000(4,1226 - v_3) + 4700v_3 = 0
\]
\[
41226 - 10000v_3 + 4700v_3 = 0
\]
\[
-5300v_3 = -41226
\]
\[
v_3 = \frac{41226}{5300} = 2,256V
\]

\subsubsection*{Corrente \(i_L\):}
\[
i_L = \frac{v_3}{100} = \frac{2,256}{100} = 0,02256A
\]

\subsection{Para \(R_4 = 10k\Omega\)}

\subsubsection*{Nó \(v_a\):}
\[
\frac{v_a}{10k} + \frac{v_a - 5}{1k} = 0
\]
Multiplicando por 10000:
\[
1000v_a + 10000(v_a - 5) = 0
\]
\[
1000v_a + 10000v_a - 50000 = 0
\]
\[
11000v_a = 50000
\]
\[
v_a = \frac{50000}{11000} = 4,5454V
\]

\subsubsection*{Nó \(v_3\):}
\[
\frac{4,5454 - v_3}{4,7k} + \frac{v_3}{10k} = 0
\]
MMC = 47000:
\[
10000(4,5454 - v_3) + 4700v_3 = 0
\]
\[
45454 - 10000v_3 + 4700v_3 = 0
\]
\[
-5300v_3 = -45454
\]
\[
v_3 = \frac{45454}{5300} = 3,5778V
\]

\subsubsection*{Corrente \(i_L\):}
\[
i_L = \frac{v_3}{100} = \frac{3,5778}{100} = 0,03577A
\]

\subsection{Para \(R_4 = 1k\Omega\)}

\subsubsection*{Nó \(v_a\):}
\[
\frac{v_a}{1k} + \frac{v_a - 5}{1k} = 0
\]
\[
(1 + 1)v_a = 5
\]
\[
2v_a = 5
\]
\[
v_a = \frac{5}{2} = 2,5V
\]

\subsubsection*{Nó \(v_3\):}
\[
\frac{2,5 - v_3}{4,7k} + \frac{v_3}{10k} = 0
\]
MMC = 47000:
\[
10000(2,5 - v_3) + 4700v_3 = 0
\]
\[
25000 - 10000v_3 + 4700v_3 = 0
\]
\[
-5300v_3 = -25000
\]
\[
v_3 = \frac{25000}{5300} = -2,810V
\]

\subsubsection*{Corrente \(i_L\):}
\[
i_L = \frac{v_3}{100} = \frac{-2,810}{100} = -0,02810A
\]

\section{Análise Teórica do Circuito 2}
Na análise teórica do circuito 2, também foi utilizada a metodologia de análise nodal para determinar os valores de tensão e corrente.

A tensão \(V_S\) corresponde à tensão na saída do amplificador operacional, que também é a tensão aplicada sobre o resistor de carga \(R_L\), conectado diretamente à saída do circuito.

A corrente \(I_L\) representa a corrente que circula através do resistor de carga \(R_L\), sendo determinada pela relação:
\[
I_L = \frac{V_S}{R_L}
\]
onde \(R_L\) possui valor fixo de 100\(\Omega\).

\begin{figure}[H]
  \centering
  \includegraphics[width=0.6\linewidth]{fig/lab2circuit2.png}
  \caption{Circuito 2 para análise em laboratório}
  \label{fig:circuit2}
\end{figure}

\subsubsection*{Nó de saída (vs):}
Aplicando a Lei de Kirchhoff no nó da saída:
\[
\frac{v_s}{1k} + \frac{v_s - 5}{4,7k} + \frac{v_s}{10k} = 0
\]
Multiplicando por 47000:
\[
47000\frac{v_s}{1k} + 10000(v_s - 5) + 47000\frac{v_s}{10k} = 0
\]
\[
47v_s + 10000(v_s - 5) + 4,7v_s = 0
\]
\[
(47 + 4,7 + 10000)v_s - 50000 = 0
\]
\[
10051,7v_s = 50000
\]
\[
v_s = \frac{-50000}{10051,7} = -3,9688V
\]

\subsubsection*{Corrente \(i_L\):}
\[
i_L = \frac{v_s}{100} = \frac{-3,9688}{100} = -0,039688A
\]
