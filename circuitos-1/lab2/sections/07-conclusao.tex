\section{Conclusão}

Com base nos resultados obtidos, conclui-se que a simulação computacional se mostra uma ferramenta extremamente útil para a análise teórica de circuitos, permitindo validar conceitos, prever comportamentos e antecipar resultados. No entanto, ela trabalha com modelos idealizados, que desconsideram as imperfeições e limitações físicas presentes no mundo real.

Por outro lado, a realização do experimento prático em laboratório evidencia a influência de fatores como tolerâncias dos componentes, ruídos, limitações dos dispositivos (como amplificadores operacionais), variações na tensão de alimentação, além de erros associados aos instrumentos de medição. Esses efeitos, embora muitas vezes desprezados na fase de projeto teórico, são determinantes para o funcionamento real dos circuitos.

Apesar dessas diferenças, observou-se que os resultados obtidos nas três abordagens — teórica, simulada e experimental — apresentaram boa concordância entre si, com uma margem de erro dentro do esperado, validando tanto os modelos desenvolvidos quanto a metodologia aplicada. Isso reforça que a simulação é uma etapa fundamental no desenvolvimento de projetos, especialmente para validar conceitos e antecipar possíveis ajustes. Contudo, é indispensável realizar a verificação prática, uma vez que apenas os testes experimentais permitem observar os efeitos das não idealidades, garantindo assim a precisão e a confiabilidade do circuito no seu funcionamento real.

Portanto, a integração entre análise teórica, simulação e experimentação prática é essencial para o desenvolvimento eficiente e robusto de projetos em eletrônica, proporcionando uma compreensão completa dos circuitos e preparando o projetista para lidar com os desafios e limitações do mundo real.
