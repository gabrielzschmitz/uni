\section{Análise dos Resultados}

Após a realização das análises nas três modalidades — teórica, simulação e experimento real —, observou-se que os valores obtidos foram, de modo geral, bastante próximos entre si, o que indica uma boa coerência dos resultados. Essa consistência valida tanto os cálculos realizados quanto os modelos simulados. No entanto, algumas discrepâncias entre os resultados foram observadas e podem ser explicadas por diferentes fatores, tanto práticos quanto teóricos.

No \textbf{Circuito 1}, um aspecto importante identificado foi a necessidade de ajustar a tensão da fonte que alimentava o amplificador operacional para que os resultados obtidos experimentalmente fossem congruentes com os valores esperados da análise teórica e da simulação. Especificamente, foi necessário variar a tensão de alimentação entre 4V e 12V, dependendo do valor adotado para o resistor \(R_4\). Esse ajuste foi essencial devido às limitações físicas dos amplificadores operacionais, que não conseguem operar corretamente se a tensão de saída se aproxima dos limites da alimentação. Enquanto na análise teórica e na simulação os amplificadores são considerados ideais (sem limitações de saturação, corrente de saída ou tensão de alimentação), no experimento real esses fatores influenciam diretamente o comportamento do circuito.

Além disso, ao inserir na simulação do MATLAB os valores reais dos componentes utilizados no experimento — considerando as variações medidas, como resistores com pequenas diferenças em relação aos seus valores nominais —, os resultados simulados apresentaram uma taxa de erro de até \textbf{5\%} em relação às simulações feitas com os valores teóricos ideais. Essa diferença é justificada principalmente pelas tolerâncias dos componentes (que, embora pequenas, impactam diretamente nas tensões e correntes), além das limitações físicas do próprio amplificador operacional, como offset de entrada, corrente de polarização, resistência de saída e limitação de faixa linear de operação.

É importante destacar que tanto os cálculos teóricos quanto os modelos simulados assumem um circuito idealizado, no qual os resistores possuem exatamente seus valores nominais, as fontes são ideais e os fios e conexões possuem resistência, capacitância e indutância desprezíveis. Já no experimento real, todos esses fatores físicos — inevitáveis na prática — afetam os resultados. Além disso, variáveis externas como ruído eletromagnético, interferências do ambiente, imprecisão dos instrumentos de medição e até variações térmicas (que podem alterar ligeiramente a resistência dos componentes) também contribuem para pequenas discrepâncias observadas.

Somam-se ainda aos fatores práticos as limitações associadas aos métodos numéricos das simulações. Embora o MATLAB utilize métodos altamente precisos, ele ainda está sujeito a erros de arredondamento, discretização e precisão de ponto flutuante, o que pode gerar pequenas diferenças nos resultados, especialmente quando o sistema é muito sensível às variações de parâmetros.

Portanto, a análise dos resultados demonstra que, apesar das pequenas variações, todos os métodos — teórico, simulado e experimental — apresentaram valores consistentes entre si, dentro de uma margem de erro aceitável, o que valida o modelo proposto. Além disso, o processo evidencia a importância de se considerar as limitações dos componentes e dos instrumentos quando se faz a transição de um modelo ideal para um circuito real, reforçando a compreensão prática dos conceitos teóricos aplicados na análise de circuitos.
