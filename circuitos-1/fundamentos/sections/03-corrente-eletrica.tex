\section{Corrente Elétrica}

Corrente elétrica é o fluxo de carga por unidade de tempo, medido em ampères
(A). Matematicamente, a relação entre a corrente \textit{i}, a carga \textit{q}
e o tempo \textit{t} é:

\begin{equation}
	\label{eq:corrente}
	i \overset{\triangle}{=} \frac{dq}{dt}
\end{equation}

Onde a corrente é medida em ampères (A) e

\[
	1\,\text{ampère} = 1\,\text{coulomb}/\text{segundo}
\]

A carga transferida entre o instante \textit{\( t_0 \)} e o instante \textit{t} é
obtida integrando os dois lados da equação~\ref{eq:corrente}, onde obtemos:

\begin{equation}
	\label{eq:carga-eletrica}
	Q \overset{\triangle}{=} \int_{t_0}^{t} i\,dt
\end{equation}

\subsection{\textbf{Exemplos}}

\begin{enumerate}
	\item Qual é a quantidade de carga representada por 4600 elétrons?
	      \[
		      -1.602 \times 10^{-19} \cdot 4600 = −7.3692 \times 10^{-16}
	      \]

	\item Calcule a quantidade de carga representada por seis milhões de prótons.
	      \[
		      -1.602 \times 10^{-19} \cdot 6000000 = −9.612 \times 10^{-13}
	      \]

	\item A carga total entrando em um terminal é dada por \( q = 5t \sin{4\pi t}
	      \) mC. Calcule a corrente no instante \( t = 0.5 \) s.
	      \[
		      \begin{aligned}
			      i & = \frac{dq}{dt}                                           \\
			        & = \frac{d}{dt}\left(5t \sin{4\pi t}\right) \, \text{mC/s} \\
			        & \hphantom{=} \diamond (fg)' =  f'g + fg'                  \\
			        & = (5 \sin{4\pi t} + 20\pi t \cos{4\pi t}) \, \text{mA}    \\
			        & \hphantom{=} \text{para } t = 0.5                         \\
			        & = (5 \sin{2\pi} + 10\pi \cos{2\pi}) \, \text{mA}          \\
			        & = 31.415 \, \text{mA}                                     \\
		      \end{aligned}
	      \]
	\item A carga total entrando em um terminal é dada por \( q = 10 -
	      10e^{-2t} \) mC. Calcule a corrente no instante \( t = 1.0 \) s.
	      \[
		      \begin{aligned}
			      i & = \frac{dq}{dt}                                          \\
			        & = \frac{d}{dt}\left(10 - 10e^{-2t}\right) \, \text{mC/s} \\
			        & = (20e^{-2t}) \, \text{mA}                               \\
			        & \hphantom{=} \text{para } t = 1.0                        \\
			        & = (20e^{-2}) \, \text{mA}                                \\
			        & = 2.706705665 \, \text{mA}                               \\
		      \end{aligned}
	      \]
	\item Determine a carga total que entra em um terminal entre os instantes \( t
	      = 1 \) s e \( t = 2 \) s se a corrente que passa pelo terminal é \( i = (3t^2
	      – t) \) A.
	      \[
		      \begin{aligned}
			      Q & = \int_{t_0}^{t} i\,dt                            \\
			        & = \int_{1}^{2} (3t^2 - t)\,dt                     \\
			        & = \left.\left(t^3-\frac{t^2}{2}\right)\right|_1^2 \\
			        & = (8 - 2) - (1 - \frac{1}{2})                     \\
			        & = 5.5 \text{C}                                    \\
		      \end{aligned}
	      \]
	\item A corrente que fui através de um elemento é
	      \[
		      \begin{aligned}
			      i = \begin{cases}4 \text{~A},     & 0<t<1 \\
             4 t^2 \text{~A}, & t>1\end{cases} \\
		      \end{aligned}
	      \]
	      Calcule a carga que entra no elemento de \( t = 0 \) a \( t = 2 \) s.
	      \[
		      \begin{aligned}
			      Q & = \int_{0}^{2} i\,dt                                           \\
			        & = \int_{0}^{1} 4\,dt + \int_{1}^{2} 4t^2\,dt                   \\
			        & = \left[4t\right]_0^1 + \left[\frac{4t^3}{3}\right]_1^2        \\
			        & = (4(1) - 4(0)) + \left(\frac{4(8)}{3} - \frac{4(1)}{3}\right) \\
			        & = 4 + \left(\frac{32}{3} - \frac{4}{3}\right)                  \\
			        & = 4 + \frac{28}{3}                                             \\
			        & = \frac{12}{3} + \frac{28}{3}                                  \\
			        & = \frac{40}{3} \text{C} \approx 13.33333333 \,\text{C}
		      \end{aligned}
	      \]
\end{enumerate}

\subsection{\textbf{Corrente Contínua e Alternada}}

Se a corrente não muda com o tempo e permanece constante, podemos chamá-la
\textit{corrente contínua} (CC). E por convenção usa-se o símbolo \textit{I}
para representa-la.

Já se a corrente muda com o tempo, podemos chamá-la \textit{corrente alternada}
(CA). E usa-se o símbolo \textit{i} para representa-la.
